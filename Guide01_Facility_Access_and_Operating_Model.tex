% Standalone LaTeX file (Overleaf-ready)
% Guide 01: Facility Access and Operating Model
% Compile with: PDFLaTeX

\documentclass[11pt]{article}

\usepackage[margin=1in]{geometry}
\usepackage{microtype}
\usepackage{xcolor}
\usepackage{hyperref}
\usepackage{booktabs}
\usepackage{longtable}
\usepackage{array}
\usepackage{enumitem}
\usepackage{siunitx}
\usepackage{fancyhdr}

\definecolor{darkblue}{rgb}{0.0, 0.0, 0.5}
\hypersetup{
  colorlinks=true,
  linkcolor=darkblue,
  urlcolor=darkblue,
  citecolor=darkblue,
  pdftitle={Guide 01: Facility Access and Operating Model},
  pdfauthor={Open Research Draft}
}

\sisetup{detect-all=true}

\newcommand{\ProgramName}{Layer--Score--Layer Neuromorphic Tile Program}
\newcommand{\GuideID}{Guide 01}
\newcommand{\GuideTitle}{Facility Access and Operating Model}
\newcommand{\DocVersion}{v1.1 (standalone)}
\newcommand{\DocDate}{\today}

\setlength{\parindent}{0pt}
\setlength{\parskip}{0.55em}

\pagestyle{fancy}
\fancyhf{}
\lhead{\GuideID: \GuideTitle}
\rhead{\DocVersion}
\cfoot{\thepage}

\newcolumntype{P}[1]{>{\raggedright\arraybackslash}p{#1}}

\begin{document}

\begin{titlepage}
  \centering
  \vspace*{1.0in}
  {\LARGE \GuideID: \GuideTitle\par}
  \vspace{0.25in}
  {\Large \ProgramName\par}
  \vspace{0.35in}
  {\large \DocVersion\par}
  \vspace{0.10in}
  {\large \DocDate\par}
  \vfill
  \begin{minipage}{0.93\textwidth}
  \small
  \textbf{Purpose.} Define a practical, step-by-step plan for gaining access to micro/nanofabrication facilities and operating them efficiently for the program. It covers: choosing an operating mode (hands-on vs staff-assisted vs remote), selecting facilities, preparing an outreach packet, onboarding, training strategy, budgeting and scheduling, and day-to-day operational discipline.
  \\
  \\
  \textbf{Safety note.} This guide is about using professional/shared facilities and commercial equipment safely and compliantly. It is not a manual for building hazardous equipment.
  \end{minipage}
  \vspace*{0.6in}
\end{titlepage}

\tableofcontents
\newpage

\section{At a Glance}

\subsection{Who this guide is for}
This guide assumes you:
\begin{itemize}[leftmargin=*,itemsep=2pt]
\item have strong electronics lab skills (instrumentation, probing, ESD discipline),
\item are comfortable with bachelor-level physics/circuits/materials concepts,
\item have access to advanced software assistance and adequate funding,
\item are new (or not expert) in cleanroom microfabrication operations and facility culture.
\end{itemize}

\subsection{What this guide produces (deliverables)}
By completing this guide, you should have:
\begin{enumerate}[leftmargin=*,itemsep=2pt]
\item \textbf{Operating Mode Decision Memo (1 page).} Which mode you are using and why.
\item \textbf{Facility Shortlist (3--6 candidates).} With a scored comparison matrix.
\item \textbf{Tool Requirement Matrix.} Must-have vs nice-to-have tools mapped to your near-term process.
\item \textbf{Outreach Packet.} A small set of documents you send to facilities and potential collaborators.
\item \textbf{Onboarding Checklist.} Accounts, agreements, safety training, tool qualifications, and access logistics.
\item \textbf{Budget and Cadence Plan.} A realistic monthly burn estimate and run cadence.
\item \textbf{Facility Q\&A Script.} A consistent set of questions to avoid surprises (especially contamination rules).
\end{enumerate}

\subsection{Why this guide matters}
Most first-time external users fail for non-technical reasons:
\begin{itemize}[leftmargin=*,itemsep=2pt]
\item they pick a facility without the required tools \emph{or} without the required materials policies,
\item they underestimate onboarding and training time,
\item they propose ``big vision'' instead of a small, facility-friendly first run,
\item they bring restricted materials (e.g., mobile metals) without a contamination plan,
\item they do not budget for rework, metrology, and packaging.
\end{itemize}

\section{Cleanroom Access in Plain Language}

\subsection{What a micro/nanofab facility is (layperson summary)}
A micro/nanofabrication facility (``nanofab'' or ``cleanroom'') is a controlled environment where you build tiny structures on wafers or chips by repeatedly:
\begin{itemize}[leftmargin=*,itemsep=2pt]
\item depositing thin films (metals and dielectrics),
\item patterning those films (lithography),
\item removing material (etch) or removing resist/metal (lift-off),
\item repeating to form multilayer stacks.
\end{itemize}

For your program, the facility is not optional: thin-film quality, patterning, and planarity determine whether a diffusive-memristor device behaves reproducibly.

\subsection{What ``external user'' means}
Most major nanofabs support external users:
\begin{itemize}[leftmargin=*,itemsep=2pt]
\item \textbf{Academic users:} university labs, PIs, students.
\item \textbf{Industry/startup users:} companies paying cost-recovery rates.
\item \textbf{Sponsored/partner users:} funded collaborations where staff assistance is part of the agreement.
\end{itemize}

External use typically requires:
\begin{itemize}[leftmargin=*,itemsep=2pt]
\item a facility use agreement,
\item safety training and tool training,
\item billing setup,
\item compliance with tool-specific contamination rules.
\end{itemize}

\subsection{Why facilities care about contamination (especially mobile metals)}
Facilities protect expensive tools and other users' work. Some materials are restricted because tiny traces can:
\begin{itemize}[leftmargin=*,itemsep=2pt]
\item poison subsequent processes,
\item degrade yield for sensitive CMOS users,
\item create difficult-to-diagnose reliability problems.
\end{itemize}

\textbf{Practical implication:} your first facility conversations should focus less on ``how cool the neuron is'' and more on ``can we safely run these materials in your toolchain.''

\subsection{New critical question (mobile-metal handling in shared vacuum tools)}
Even if you plan to encapsulate mobile metals, many facilities treat encapsulation as risk reduction rather than risk elimination. Therefore add this non-negotiable question to your first call:
\begin{quote}
\emph{Do you accept encapsulated mobile metals in shared vacuum tools, or must we use a segregated toolset end-to-end for the entire flow?}
\end{quote}

\section{Step-by-Step Procedure}

\subsection{Step 1: Choose your operating mode (decision tree)}
\textbf{Goal:} Decide how you will actually run fabrication, based on your experience, schedule, and risk tolerance.

\subsubsection{1.1 The four realistic modes}
\begin{enumerate}[leftmargin=*,itemsep=4pt]
\item \textbf{Mode A: You operate tools yourself (trained user).}\\
Pros: maximum control, fastest iteration once trained, lower long-run cost.\\
Cons: training time is real; tool access and competency gates can take months; mistakes are costly.

\item \textbf{Mode B: Staff-assisted use (facility staff runs critical steps).}\\
Pros: faster initial progress; fewer operator mistakes; better for complex tools (ALD, CMP, high-resolution tools).\\
Cons: higher cost; scheduling dependency.

\item \textbf{Mode C: Remote ``foundry model'' (you deliver designs; staff executes).}\\
Pros: fastest onboarding for non-experts; professional execution; lower personal training burden.\\
Cons: less flexibility for rapid changes; requires strong documentation and communication.

\item \textbf{Mode D: Partner with a PI/lab (sponsored research or hired staff).}\\
Pros: facility access is native; lab has process intuition; staffing scales; highest probability of success.\\
Cons: requires relationship management; IP/publication expectations must be clarified; timelines depend on people.
\end{enumerate}

\subsubsection{1.2 Recommended operating mode for this program (early phase)}
For a well-funded electronics-lab operator who is new to microfab operations, the highest probability approach is:
\begin{itemize}[leftmargin=*,itemsep=2pt]
\item Start in \textbf{Mode B or C} for the first 2--4 fabrication iterations,
\item transition to \textbf{Mode A} for a narrow set of safe, repeatable tools,
\item consider \textbf{Mode D} if you want to scale to more complex integration quickly.
\end{itemize}

\subsubsection{1.3 Make the decision explicit}
Write a 1-page decision memo:
\begin{itemize}[leftmargin=*,itemsep=2pt]
\item chosen mode (A/B/C/D),
\item why you chose it,
\item what would make you switch modes,
\item what tools/steps you will never do without staff assistance (at least initially).
\end{itemize}

\textbf{Output:} Operating Mode Decision Memo v1.0.

\subsection{Step 2: Translate your near-term device plan into tool requirements}
\textbf{Goal:} Create a short list of must-have tools and nice-to-have tools, mapped to a minimal viable process.

\subsubsection{2.1 Start with a minimal viable device objective}
For facility onboarding, do \emph{not} pitch ``full tile'' as the first run. Pitch something facility-friendly:
\begin{itemize}[leftmargin=*,itemsep=2pt]
\item \textbf{First-run objective:} Fabricate simple diffusive-memristor test structures (e.g., MIM stacks) and characterize volatility and threshold statistics.
\end{itemize}

\subsubsection{2.2 Define your ``Must-have tool'' set}
A typical must-have set for early diffusive-memristor work:
\begin{itemize}[leftmargin=*,itemsep=2pt]
\item \textbf{Deposition:} PVD metal deposition.
\item \textbf{Active dielectric deposition:} ALD is recommended as a default for the active switching dielectric.
\item \textbf{Patterning:} maskless lithography or photolithography; liftoff capability.
\item \textbf{Etch/clean:} plasma cleaning and/or RIE suitable for the dielectric.
\item \textbf{Metrology:} optical inspection; thickness measurement.
\item \textbf{Electrical test:} probe station access.
\end{itemize}

\subsubsection{2.3 Define your ``Nice-to-have'' tool set}
Useful later:
\begin{itemize}[leftmargin=*,itemsep=2pt]
\item CMP planarization,
\item EBL or FIB for localized scoring/repair,
\item advanced microscopy,
\item packaging support.
\end{itemize}

\subsubsection{2.4 Create a one-page tool requirement matrix}
Use Table~\ref{tab:toolmatrix} template and fill it.

\begin{longtable}{@{}P{3.0cm}P{5.0cm}P{3.0cm}P{4.0cm}@{}}
\caption{Tool Requirement Matrix Template (fill for each facility).}
\label{tab:toolmatrix}\\
\toprule
\textbf{Process Need} & \textbf{Tool Type} & \textbf{Must / Nice} & \textbf{Facility Notes} \\
\midrule
\endfirsthead
\toprule
\textbf{Process Need} & \textbf{Tool Type} & \textbf{Must / Nice} & \textbf{Facility Notes} \\
\midrule
\endhead
Metal electrodes & Sputter / evaporation & Must & allowed metals; toolset segregation \\
Active dielectric film & ALD dielectric & Must & precursor restrictions; thermal budget \\
Patterning & Maskless lithography / stepper & Must & min feature; overlay \\
Pattern transfer & Liftoff; RIE; wet etch & Must & etch chemistries; compatibility \\
Cleaning & Plasma cleaning & Must & contamination classes \\
Planarization & CMP & Nice & availability; staff-assisted? \\
Fine ``scoring'' & EBL / FIB / equivalent & Nice & ROI limitation; scheduling \\
Metrology & Optical, profilometer, ellipsometer & Must & access policy \\
Electrical test & Probe station & Must & who can probe; allowed voltages \\
Packaging & Dicing / wirebond & Nice & staff assistance; cost \\
\bottomrule
\end{longtable}

\textbf{Output:} Tool Requirement Matrix v1.0.

\subsection{Step 3: Build a facility shortlist (3--6 candidates)}
\textbf{Goal:} Avoid falling in love with the first facility you find.

\subsubsection{3.1 Define selection criteria}
Score each facility on:
\begin{itemize}[leftmargin=*,itemsep=2pt]
\item \textbf{Tool coverage,}
\item \textbf{Materials policy fit} (including whether encapsulated mobile metals can enter shared vacuum tools),
\item \textbf{Operating mode support} (staff-assisted/remote),
\item \textbf{Cadence feasibility,}
\item \textbf{Cost transparency,}
\item \textbf{Metrology/testing,}
\item \textbf{Onboarding speed.}
\end{itemize}

\subsubsection{3.2 Use a scored comparison matrix}
Fill Table~\ref{tab:facilitymatrix}. Use a 1--5 scale (1 = poor, 5 = excellent).

\begin{longtable}{@{}P{3.2cm}P{1.2cm}P{1.2cm}P{1.2cm}P{1.2cm}P{1.2cm}P{1.2cm}P{3.2cm}@{}}
\caption{Facility Shortlist Scoring Matrix Template.}
\label{tab:facilitymatrix}\\
\toprule
\textbf{Facility} & \textbf{Tools} & \textbf{Policy} & \textbf{Mode} & \textbf{Cadence} & \textbf{Cost} & \textbf{Metrology} & \textbf{Notes} \\
\midrule
\endfirsthead
\toprule
\textbf{Facility} & \textbf{Tools} & \textbf{Policy} & \textbf{Mode} & \textbf{Cadence} & \textbf{Cost} & \textbf{Metrology} & \textbf{Notes} \\
\midrule
\endhead
Facility A &  &  &  &  &  &  &  \\
Facility B &  &  &  &  &  &  &  \\
Facility C &  &  &  &  &  &  &  \\
Facility D &  &  &  &  &  &  &  \\
\bottomrule
\end{longtable}

\textbf{Output:} Facility Shortlist v1.0.

\subsection{Step 4: Build the outreach packet (what you send to facilities)}
\textbf{Goal:} Make it easy for facilities to say ``yes.''

\subsubsection{4.1 Packet contents (recommended)}
Keep it short and professional:
\begin{enumerate}[leftmargin=*,itemsep=2pt]
\item \textbf{One-page project brief.}
\item \textbf{Materials disclosure sheet.}
\item \textbf{Process outline (high-level).}
\item \textbf{Tool requirement matrix.}
\item \textbf{Operating mode request.}
\item \textbf{Safety/contamination posture.}
\item \textbf{Scheduling and budget intent.}
\end{enumerate}

\subsubsection{4.2 How to phrase the first-run objective}
Facilities respond well to small, concrete objectives:
\begin{itemize}[leftmargin=*,itemsep=2pt]
\item ``We want to fabricate MIM test devices to measure volatile switching and statistical variability.''
\item ``We will start with a small number of wafers/dies and publish a reproducibility dataset.''
\end{itemize}

\textbf{Output:} Outreach Packet v1.0.

\subsection{Step 5: Run the first facility call (script + agenda)}
\textbf{Goal:} Identify deal-breakers and define onboarding.

\subsubsection{5.1 Suggested agenda (30--45 minutes)}
\begin{enumerate}[leftmargin=*,itemsep=2pt]
\item Introductions and first-run objective.
\item Tool capability fit.
\item Materials/contamination policy fit.
\item Operating mode options.
\item Administrative onboarding steps.
\item Next steps.
\end{enumerate}

\subsubsection{5.2 The critical questions to ask (non-negotiable)}
\textbf{Materials and contamination}
\begin{itemize}[leftmargin=*,itemsep=2pt]
\item Are Ag-bearing samples allowed? Under what restrictions?
\item \textbf{Do you accept encapsulated mobile metals in shared vacuum tools, or must we use segregated tools end-to-end?}
\item Which toolsets are dedicated/segregated?
\item Is encapsulation required before specific tools? What qualifies?
\end{itemize}

\textbf{Tools and process}
\begin{itemize}[leftmargin=*,itemsep=2pt]
\item Do you have ALD suitable for our active dielectric?
\item Is maskless lithography available to external users? Typical overlay?
\item Do you have RIE/plasma cleaning suitable for our dielectric system?
\item Do you have CMP or planarization options?
\end{itemize}

\textbf{Operating model and administrative}
\begin{itemize}[leftmargin=*,itemsep=2pt]
\item Can staff run steps for external users? Billing model?
\item Is remote fabrication supported?
\item Agreements, billing, lead times, and scheduling constraints?
\end{itemize}

\textbf{Output:} Facility Call Notes + Action Items.

\subsection{Step 6: Complete onboarding (accounts, agreements, safety)}
Common components:
\begin{itemize}[leftmargin=*,itemsep=2pt]
\item registration, agreements, insurance,
\item safety training,
\item tool training and qualification,
\item badge and scheduling access.
\end{itemize}

\textbf{Output:} Onboarding Checklist.

\subsection{Step 7: Build a realistic budget and cadence plan}
Budget categories:
\begin{itemize}[leftmargin=*,itemsep=2pt]
\item onboarding/training fees,
\item tool hourly charges and staff recharge,
\item consumables and materials,
\item metrology time,
\item packaging/dicing,
\item travel,
\item contingency for rework/failed runs.
\end{itemize}

Define what a ``run'' is and set an early cadence (often monthly runs + weekly analysis cycles).

\textbf{Output:} Budget + Cadence Plan.

\subsection{Step 8: Establish day-to-day facility operating discipline}
\subsubsection{8.1 Pre-run readiness checklist (minimum)}
\begin{itemize}[leftmargin=*,itemsep=2pt]
\item design frozen and version-tagged,
\item traveler reviewed,
\item materials confirmed,
\item sample state and segregation plan approved,
\item measurement plan ready,
\item backup plan for tool downtime.
\end{itemize}

\subsubsection{8.2 Tool day checklist (minimum)}
\begin{itemize}[leftmargin=*,itemsep=2pt]
\item before/after photos,
\item record recipe IDs and parameters,
\item log deviations immediately,
\item label and store samples in correct state.
\end{itemize}

\subsubsection{8.3 Post-run checklist (minimum)}
\begin{itemize}[leftmargin=*,itemsep=2pt]
\item upload raw data same day,
\item regenerate plots by script,
\item write run summary,
\item update compact model if applicable.
\end{itemize}

\textbf{Output:} Facility Operating SOP v1.0.

\section{Common Pitfalls and How to Avoid Them}
\subsection{Pitfall A: ``We need EBL for everything''}
If you depend on EBL/FIB as the main patterning method, throughput collapses.\
\textbf{Mitigation:} maskless lithography is the backbone; scoring is ROI-limited.

\subsection{Pitfall B: ``Encapsulation makes mobile metals universally safe''}
Encapsulation reduces risk but does not eliminate it.\
\textbf{Mitigation:} ask the explicit shared-vacuum question; build a segregated flow if required.

\subsection{Pitfall C: ``We can skip metrology to save money''}
Skipping metrology creates mystery failures.\
\textbf{Mitigation:} mandatory metrology checkpoints.

\subsection{Pitfall D: ``We will document later''}
Later never comes.\
\textbf{Mitigation:} travelers, deviation logs, and script-generated plots from day one.

\section{Exit Criteria: When you are done with Guide 01}
You have completed Guide 01 when:
\begin{enumerate}[leftmargin=*,itemsep=2pt]
\item operating mode is selected and documented,
\item shortlist of candidate facilities is scored,
\item at least one facility/toolchain can execute the first-run objective without policy contradictions after Ag introduction,
\item outreach packet exists,
\item onboarding checklist exists with owners/dates,
\item budget/cadence supports at least three iterations.
\end{enumerate}

\end{document}
