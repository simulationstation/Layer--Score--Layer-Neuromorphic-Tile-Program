% Standalone LaTeX file (Overleaf-ready)
% Guide 08: Planarization Strategy
% Compile with: PDFLaTeX

\documentclass[11pt]{article}

\usepackage[margin=1in]{geometry}
\usepackage{microtype}
\usepackage{xcolor}
\usepackage{hyperref}
\usepackage{booktabs}
\usepackage{longtable}
\usepackage{array}
\usepackage{enumitem}
\usepackage{siunitx}
\usepackage{fancyhdr}

\definecolor{darkblue}{rgb}{0.0, 0.0, 0.5}
\hypersetup{colorlinks=true, linkcolor=darkblue, urlcolor=darkblue, citecolor=darkblue, pdftitle={Guide 08: Planarization Strategy}, pdfauthor={Open Research Draft}}
\sisetup{detect-all=true}

\newcommand{\ProgramName}{Layer--Score--Layer Neuromorphic Tile Program}
\newcommand{\GuideID}{Guide 08}
\newcommand{\GuideTitle}{Planarization Strategy}
\newcommand{\DocVersion}{v1.1 (standalone)}
\newcommand{\DocDate}{\today}

\setlength{\parindent}{0pt}
\setlength{\parskip}{0.55em}

\pagestyle{fancy}
\fancyhf{}
\lhead{\GuideID: \GuideTitle}
\rhead{\DocVersion}
\cfoot{\thepage}

\begin{document}

\begin{titlepage}
  \centering
  \vspace*{1.0in}
  {\LARGE \GuideID: \GuideTitle\par}
  \vspace{0.25in}
  {\Large \ProgramName\par}
  \vspace{0.35in}
  {\large \DocVersion\par}
  \vspace{0.10in}
  {\large \DocDate\par}
  \vfill
  \begin{minipage}{0.93\textwidth}
  \small
  \textbf{Purpose.} Prevent overlay and focus collapse in multilayer stacks by explicitly managing topography and planarity. This guide defines the planarity approach, metrology checkpoints, acceptance limits, and rework policies.
  \end{minipage}
  \vspace*{0.6in}
\end{titlepage}

\tableofcontents
\newpage

\section{Core Deliverables}
\begin{enumerate}[leftmargin=*,itemsep=2pt]
\item Chosen planarity path (CMP, planarizing dielectric + etch-back, or low-topography design constraints)
\item Metrology plan for topography (profilometry/AFM sites and sampling plan)
\item Maximum allowable step-height policy before the next lithography step
\item Rework policy and escalation rules
\item Reporting requirements (planarity metrics included in every run summary)
\end{enumerate}

\section{Step-by-Step Procedure}
\subsection{Step 1: Choose your planarity approach}
\begin{enumerate}[leftmargin=*,itemsep=2pt]
\item Decide your planarity method for the next two to three runs and document it.
\item If CMP is used, define whether it is staff-assisted or operator-run, and define the measurement gate after CMP.
\item If dielectric + etch-back is used, define thickness targets and etch-back stopping criteria.
\item If low-topography constraints are used, document design constraints (layer count limits, routing constraints, max metal thickness).
\end{enumerate}

\subsection{Step 2: Define metrology sites and cadence}
\begin{enumerate}[leftmargin=*,itemsep=2pt]
\item Choose standard sites (per die/wafer) where step height and thickness will be measured.
\item Define the cadence: after each major deposition, after each pattern transfer, and after any planarization step.
\item Record all measurements in the traveler (Guide 04) with tool IDs and timestamps.
\end{enumerate}

\subsection{Step 3: Define acceptance limits and rework policy}
\begin{enumerate}[leftmargin=*,itemsep=2pt]
\item Define a maximum allowable step height before the next lithography step.
\item Define what happens when the limit is exceeded: planarize, redesign, or stop.
\item Define escalation rules: when a failure triggers staff review or a program pivot.
\end{enumerate}

\section{Exit Criteria}
Guide 08 is complete when multilayer runs show stable alignment and pattern quality without unexplained failures attributable to topography.

\end{document}
