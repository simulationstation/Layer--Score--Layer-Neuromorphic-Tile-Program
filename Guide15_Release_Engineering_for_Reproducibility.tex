% Standalone LaTeX file (Overleaf-ready)
% Guide 15: Release Engineering for Open-Source Reproducibility
% Compile with: PDFLaTeX

\documentclass[11pt]{article}

\usepackage[margin=1in]{geometry}
\usepackage{microtype}
\usepackage{xcolor}
\usepackage{hyperref}
\usepackage{booktabs}
\usepackage{longtable}
\usepackage{array}
\usepackage{enumitem}
\usepackage{siunitx}
\usepackage{fancyhdr}

\definecolor{darkblue}{rgb}{0.0, 0.0, 0.5}
\hypersetup{
  colorlinks=true,
  linkcolor=darkblue,
  urlcolor=darkblue,
  citecolor=darkblue,
  pdftitle={Guide 15: Release Engineering for Open-Source Reproducibility},
  pdfauthor={Open Research Draft}
}

\sisetup{detect-all=true}

\newcommand{\ProgramName}{Layer--Score--Layer Neuromorphic Tile Program}
\newcommand{\GuideID}{Guide 15}
\newcommand{\GuideTitle}{Release Engineering for Open-Source Reproducibility}
\newcommand{\DocVersion}{v1.1 (standalone)}
\newcommand{\DocDate}{\today}

\setlength{\parindent}{0pt}
\setlength{\parskip}{0.55em}

\pagestyle{fancy}
\fancyhf{}
\lhead{\GuideID: \GuideTitle}
\rhead{\DocVersion}
\cfoot{\thepage}

\begin{document}

\begin{titlepage}
  \centering
  \vspace*{1.0in}
  {\LARGE \GuideID: \GuideTitle\par}
  \vspace{0.25in}
  {\Large \ProgramName\par}
  \vspace{0.35in}
  {\large \DocVersion\par}
  \vspace{0.10in}
  {\large \DocDate\par}
  \vfill
  \begin{minipage}{0.93\textwidth}
  \small
  \textbf{Purpose.} Ensure every claim is reproducible: tagged datasets, scripts, environment definitions, and governance. This guide defines versioning, releases, data packaging, and contributor processes.
  \end{minipage}
  \vspace*{0.6in}
\end{titlepage}

\tableofcontents
\newpage

\section{Purpose and Philosophy}
Release engineering is the enforcement mechanism for ``no data, no claim.'' The goal is that a third party can regenerate figures and metrics from a release tag without informal communication.

\section{Core Deliverables}
\begin{enumerate}[leftmargin=*,itemsep=2pt]
\item Versioning scheme (design/process/model/benchmark)
\item Release checklist and gate criteria
\item Data packaging format (raw + derived + metadata) and integrity checks
\item Contribution guidelines (schemas, review gates)
\item Governance process for changes affecting comparability (RFC workflow)
\end{enumerate}

\section{Versioning Scheme}
Define separate versions for:
\begin{itemize}[leftmargin=*,itemsep=2pt]
\item \textbf{Design} (layout revision, padout changes, test structure changes)
\item \textbf{Process} (traveler revision, tool/recipe changes, materials lot changes)
\item \textbf{Model} (parameter set revision, model code revision)
\item \textbf{Benchmark} (dataset/preprocess revision)
\end{itemize}

\section{What Constitutes a Release}
A release is a tagged set of artifacts that includes:
\begin{itemize}[leftmargin=*,itemsep=2pt]
\item a specific design revision and process revision,
\item raw datasets with metadata,
\item analysis scripts that regenerate all reported figures,
\item model code and parameter sets (if used),
\item a release note summarizing results, limitations, and environment.
\end{itemize}

\section{Release Checklist (Minimum)}
\begin{enumerate}[leftmargin=*,itemsep=2pt]
\item Tag repository commit(s) and record tag in release note.
\item Freeze raw data and metadata (read-only).
\item Validate that scripts regenerate figures from raw data.
\item Validate that environment definitions (dependencies) are complete.
\item Confirm KPIs and measurement boundaries are unchanged or explicitly versioned.
\item Publish known limitations and failure regimes.
\end{enumerate}

\section{Data Packaging and Integrity}
Recommended rules:
\begin{itemize}[leftmargin=*,itemsep=2pt]
\item Every dataset includes metadata: run ID, device ID, tool/recipe IDs, timestamps, materials lots, sample states.
\item Use deterministic filenames and directory structure (by run ID and device ID).
\item Include integrity checks (hashes) for raw data bundles.
\end{itemize}

\section{Contribution and Governance}
\subsection{Contribution rules}
\begin{itemize}[leftmargin=*,itemsep=2pt]
\item Contributions that alter schemas or KPI definitions require an RFC.
\item Data contributions must meet minimum metadata completeness.
\item Code contributions must include tests or validation notebooks where applicable.
\end{itemize}

\subsection{RFC workflow}
An RFC is required for changes that affect comparability, including:
\begin{itemize}[leftmargin=*,itemsep=2pt]
\item KPI definitions,
\item measurement boundaries,
\item data schemas,
\item benchmark selection or preprocessing.
\end{itemize}

\section{Exit Criteria}
Guide 15 is complete when external replicators can regenerate figures and metrics from release artifacts within stated uncertainty, using your published scripts and schemas.

\end{document}
