% Standalone LaTeX file (Overleaf-ready)
% Guide 07: Lift-off vs Etch Decision Tree (with Critical-Edge Fence Controls)
% Compile with: PDFLaTeX

\documentclass[11pt]{article}

\usepackage[margin=1in]{geometry}
\usepackage{microtype}
\usepackage{xcolor}
\usepackage{hyperref}
\usepackage{booktabs}
\usepackage{longtable}
\usepackage{array}
\usepackage{enumitem}
\usepackage{siunitx}
\usepackage{fancyhdr}

\definecolor{darkblue}{rgb}{0.0, 0.0, 0.5}
\hypersetup{
  colorlinks=true,
  linkcolor=darkblue,
  urlcolor=darkblue,
  citecolor=darkblue,
  pdftitle={Guide 07: Lift-off vs Etch Decision Tree},
  pdfauthor={Open Research Draft}
}

\sisetup{detect-all=true}

\newcommand{\ProgramName}{Layer--Score--Layer Neuromorphic Tile Program}
\newcommand{\GuideID}{Guide 07}
\newcommand{\GuideTitle}{Lift-off vs Etch Decision Tree (Critical-Edge Controls)}
\newcommand{\DocVersion}{v1.1 (standalone)}
\newcommand{\DocDate}{\today}

\setlength{\parindent}{0pt}
\setlength{\parskip}{0.55em}

\pagestyle{fancy}
\fancyhf{}
\lhead{\GuideID: \GuideTitle}
\rhead{\DocVersion}
\cfoot{\thepage}

\begin{document}

\begin{titlepage}
  \centering
  \vspace*{1.0in}
  {\LARGE \GuideID: \GuideTitle\par}
  \vspace{0.25in}
  {\Large \ProgramName\par}
  \vspace{0.35in}
  {\large \DocVersion\par}
  \vspace{0.10in}
  {\large \DocDate\par}
  \vfill
  \begin{minipage}{0.93\textwidth}
  \small
  \textbf{Purpose.} Standardize pattern transfer choices (lift-off versus etch) to reduce yield loss and ambiguous failure modes. This guide explicitly addresses metal edge residues (``fences'') and topography spikes as first-order hazards when thin switching dielectrics are deposited on top of patterned metals.
  \end{minipage}
  \vspace*{0.6in}
\end{titlepage}

\tableofcontents
\newpage

\section{Why This Guide Exists (Layperson Summary)}
Pattern transfer is how you turn a ``photo'' of your pattern (lithography resist) into real metal or dielectric geometry.

Two common approaches:
\begin{itemize}[leftmargin=*,itemsep=2pt]
\item \textbf{Lift-off:} deposit metal over patterned resist, then dissolve resist so unwanted metal ``lifts off.''
\item \textbf{Etch:} deposit a blanket film, pattern resist, then remove exposed film by etching.
\end{itemize}

Both can work, but they fail in different ways. For memristive devices, small edge defects can dominate switching behavior and produce misleading ``shorts/opens'' maps.

\section{Core Deliverables}
\begin{enumerate}[leftmargin=*,itemsep=2pt]
\item Decision tree: when lift-off is preferred vs when etch is preferred.
\item Inspection checklist after pattern transfer.
\item Electrical continuity verification for routing layers.
\item \textbf{Critical-edge layer identification rule and inspection gate} (mandatory).
\end{enumerate}

\section{Key Update: Lift-off Edge Artifacts (``Fences'') Are a First-Order Hazard}
When you lift off a metal, tiny ridges/residues can form at the edges of features. If you later deposit a very thin dielectric layer (the switching layer), those residues can:
\begin{itemize}[leftmargin=*,itemsep=2pt]
\item locally thin the dielectric,
\item create local electric-field hot spots,
\item short the device,
\item or cause apparent ``switching'' that is actually edge damage.
\end{itemize}

\section{Critical-Edge Layer Identification}
A metal layer is a \textbf{critical-edge layer} if it:
\begin{itemize}[leftmargin=*,itemsep=2pt]
\item sits beneath the active switching dielectric, and
\item defines the active region boundary where the dielectric will be thin.
\end{itemize}

\section{Mandatory Inspection Gate (Critical-Edge Layers)}
\textbf{Gate:} After patterning a critical-edge layer and \emph{before} depositing the thin switching dielectric:
\begin{itemize}[leftmargin=*,itemsep=2pt]
\item perform inspection at standard sites (optical minimum), and
\item perform at least one higher-magnification inspection available in your environment (facility microscopy),
\item apply acceptance criteria for edge cleanliness/topography.
\end{itemize}
Proceed only if the layer passes.

\section{Decision Guidance (High Level)}
\subsection{Prefer Lift-off When}
\begin{itemize}[leftmargin=*,itemsep=2pt]
\item etch chemistry is not available or is unsafe/incompatible,
\item etch selectivity is poor and would damage underlying layers,
\item geometry is simple and you can enforce a robust undercut profile.
\end{itemize}

\subsection{Prefer Etch When}
\begin{itemize}[leftmargin=*,itemsep=2pt]
\item edge cleanliness is critical (e.g., critical-edge metals),
\item you need tighter dimensional control,
\item multilayer stacks require repeatable sidewall profiles.
\end{itemize}

\section{Inspection Checklist (Minimum)}
After transfer (lift-off or etch):
\begin{itemize}[leftmargin=*,itemsep=2pt]
\item Check for opens/shorts on continuity structures.
\item Check for residues, edge roughness, and visible ``ears'' at metal edges.
\item Record images at standard sites and attach them to the traveler.
\item If the layer is critical-edge: apply the inspection gate and document pass/fail.
\end{itemize}

\section{Electrical Continuity Verification (Routing Layers)}
\begin{itemize}[leftmargin=*,itemsep=2pt]
\item Use predefined continuity chains and Kelvin structures.
\item Record pass/fail and measured resistance values.
\item If continuity fails, stop and determine whether failure is lithography, transfer, or film-related.
\end{itemize}

\section{Exit Criteria}
Guide 07 is complete when:
\begin{itemize}[leftmargin=*,itemsep=2pt]
\item each baseline layer has a documented default transfer method,
\item critical-edge layers are identified in the traveler,
\item critical-edge inspection gates and acceptance criteria are defined and enforced.
\end{itemize}

\end{document}
