% Standalone LaTeX file (Overleaf-ready)
% Guide 12: Array Design and Sneak-Path Management (Selector/1T1R Gate)
% Compile with: PDFLaTeX

\documentclass[11pt]{article}

\usepackage[margin=1in]{geometry}
\usepackage{microtype}
\usepackage{xcolor}
\usepackage{hyperref}
\usepackage{booktabs}
\usepackage{longtable}
\usepackage{array}
\usepackage{enumitem}
\usepackage{siunitx}
\usepackage{fancyhdr}

\definecolor{darkblue}{rgb}{0.0, 0.0, 0.5}
\hypersetup{
  colorlinks=true,
  linkcolor=darkblue,
  urlcolor=darkblue,
  citecolor=darkblue,
  pdftitle={Guide 12: Array Design and Sneak-Path Management},
  pdfauthor={Open Research Draft}
}

\sisetup{detect-all=true}

\newcommand{\ProgramName}{Layer--Score--Layer Neuromorphic Tile Program}
\newcommand{\GuideID}{Guide 12}
\newcommand{\GuideTitle}{Array Design and Sneak-Path Management}
\newcommand{\DocVersion}{v1.1 (standalone)}
\newcommand{\DocDate}{\today}

\setlength{\parindent}{0pt}
\setlength{\parskip}{0.55em}

\pagestyle{fancy}
\fancyhf{}
\lhead{\GuideID: \GuideTitle}
\rhead{\DocVersion}
\cfoot{\thepage}

\newcolumntype{P}[1]{>{\raggedright\arraybackslash}p{#1}}

\begin{document}

\begin{titlepage}
  \centering
  \vspace*{1.0in}
  {\LARGE \GuideID: \GuideTitle\par}
  \vspace{0.25in}
  {\Large \ProgramName\par}
  \vspace{0.35in}
  {\large \DocVersion\par}
  \vspace{0.10in}
  {\large \DocDate\par}
  \vfill
  \begin{minipage}{0.93\textwidth}
  \small
  \textbf{Purpose.} Prevent arrays from becoming uninterpretable due to sneak paths, line resistance, and crosstalk; define explicit gates that must be passed before scaling passive arrays for addressable compute.
  \\
  \\
  \textbf{Key policy.} Passive crossbars are \emph{characterization-first} by default. Claims of addressable scaling require a demonstrated selector strategy or a transistor backplane under the intended bias scheme.
  \end{minipage}
  \vspace*{0.6in}
\end{titlepage}

\tableofcontents
\newpage

\section{Purpose and Scope}
This guide covers:
\begin{itemize}[leftmargin=*,itemsep=2pt]
\item array architecture selection by maturity stage,
\item addressing and padout standards,
\item built-in test structures for disentangling physics from routing artifacts,
\item sneak-path diagnostics and mitigation documentation,
\item yield reporting formats and promotion gates.
\end{itemize}

\section{Core Deliverables}
\begin{enumerate}[leftmargin=*,itemsep=2pt]
\item Array architecture selection document (passive crossbar vs 1T1R/selector)
\item Addressing scheme and padout standard
\item Embedded test structures (isolated cells, line monitors, references)
\item Bring-up scripts (scan, map, classify)
\item Yield reporting format (maps, histograms, spatial correlations)
\item Sneak-path diagnostics report (measured) and mitigation plan
\end{enumerate}

\section{Policy Update: Passive Arrays Are Characterization-First}
Passive crossbars (no transistor per crosspoint) are assumed to be \textbf{characterization vehicles} unless selector/backplane behavior is proven:
\begin{itemize}[leftmargin=*,itemsep=2pt]
\item device-to-device and spatial variability mapping,
\item process learning and failure mode localization,
\item compact-model fitting inputs.
\end{itemize}

\section{Selector/Backplane Gate Before Scaling for Addressable Compute}
Before scaling beyond small arrays \emph{for addressability or compute}, you must demonstrate either:
\begin{itemize}[leftmargin=*,itemsep=2pt]
\item a transistor backplane (1T1R), or
\item a selector strategy (1S1R or intrinsic nonlinearity shown sufficient under the bias scheme).
\end{itemize}

\section{Selectivity Ratio (Formal Definition)}
A practical selectivity metric compares selected-cell current at read bias to unselected-cell current at half-select bias:

\[
\sigma \;=\; \frac{I_{\mathrm{sel}}(V_{\mathrm{read}})}{I_{\mathrm{unsel}}(V_{\mathrm{half}})}
\]

\textbf{Explainer table (every symbol defined):}

\begin{center}
\begin{tabular}{@{}P{3.2cm}P{8.7cm}P{2.0cm}@{}}
\toprule
\textbf{Symbol} & \textbf{Meaning} & \textbf{Units} \\
\midrule
$\sigma$ & Selectivity ratio under a defined addressing scheme & none \\
$I_{\mathrm{sel}}(V_{\mathrm{read}})$ & Current through the selected array cell when biased at read voltage $V_{\mathrm{read}}$ & ampere \\
$I_{\mathrm{unsel}}(V_{\mathrm{half}})$ & Current through an unselected array cell when biased at half-select voltage $V_{\mathrm{half}}$ (or analogous unselected bias) & ampere \\
$V_{\mathrm{read}}$ & Read voltage applied to the selected cell under the addressing scheme & volt \\
$V_{\mathrm{half}}$ & Half-select voltage applied to unselected cells under a half-select scheme (or analogous unselected bias) & volt \\
\bottomrule
\end{tabular}
\end{center}

\section{Step-by-Step Procedure}
\subsection{Step 1: Choose the smallest array that still teaches you something}
\begin{itemize}[leftmargin=*,itemsep=2pt]
\item Early arrays should be sized for statistics, not for performance claims.
\item Default to sizes that allow fast mapping and failure localization.
\end{itemize}

\subsection{Step 2: Build in test structures}
Include around (and within) the array:
\begin{itemize}[leftmargin=*,itemsep=2pt]
\item isolated single cells,
\item line resistance monitors,
\item reference devices,
\item continuity chains.
\end{itemize}

\subsection{Step 3: Define scan and classification protocol}
\begin{itemize}[leftmargin=*,itemsep=2pt]
\item Define what constitutes ``working,'' ``open,'' ``short,'' and ``drift.''
\item Ensure scripts log raw measurements and metadata.
\end{itemize}

\subsection{Step 4: Quantify sneak-path behavior (measured)}
\begin{itemize}[leftmargin=*,itemsep=2pt]
\item Do not treat sneak paths as purely a software problem.
\item Document the bias scheme and the measured ambiguity/noise floor.
\end{itemize}

\subsection{Step 5: If claiming addressable scaling, justify and demonstrate $\sigma_{\min}$}
\begin{itemize}[leftmargin=*,itemsep=2pt]
\item Choose a target minimum selectivity $\sigma_{\min}$ appropriate to your array size and sense method.
\item Justify it (noise floor, line resistance, expected worst-case unselected currents).
\item Demonstrate it experimentally before scaling.
\end{itemize}

\section{Exit Criteria}
Guide 12 is complete when:
\begin{itemize}[leftmargin=*,itemsep=2pt]
\item you can produce an array yield map and failure classification with evidence,
\item you can separate device-level failures from routing-level failures with test structures,
\item if you claim addressable scaling, you have a documented selector/backplane strategy or a measured $\sigma\ge\sigma_{\min}$ under the intended bias scheme.
\end{itemize}

\end{document}
