% Standalone LaTeX file (Overleaf-ready)
% Guide 00: Program Framing and Success Metrics
% Compile with: PDFLaTeX

\documentclass[11pt]{article}

\usepackage[margin=1in]{geometry}
\usepackage{microtype}
\usepackage{xcolor}
\usepackage{hyperref}
\usepackage{booktabs}
\usepackage{longtable}
\usepackage{array}
\usepackage{enumitem}
\usepackage{siunitx}
\usepackage{fancyhdr}

\definecolor{darkblue}{rgb}{0.0, 0.0, 0.5}
\hypersetup{
  colorlinks=true,
  linkcolor=darkblue,
  urlcolor=darkblue,
  citecolor=darkblue,
  pdftitle={Guide 00: Program Framing and Success Metrics},
  pdfauthor={Open Research Draft}
}

\sisetup{detect-all=true}

\newcommand{\ProgramName}{Layer--Score--Layer Neuromorphic Tile Program}
\newcommand{\GuideID}{Guide 00}
\newcommand{\GuideTitle}{Program Framing and Success Metrics}
\newcommand{\DocVersion}{v1.1 (standalone)}
\newcommand{\DocDate}{\today}

\setlength{\parindent}{0pt}
\setlength{\parskip}{0.55em}

\pagestyle{fancy}
\fancyhf{}
\lhead{\GuideID: \GuideTitle}
\rhead{\DocVersion}
\cfoot{\thepage}

\newcolumntype{P}[1]{>{\raggedright\arraybackslash}p{#1}}

\begin{document}

\begin{titlepage}
  \centering
  \vspace*{1.0in}
  {\LARGE \GuideID: \GuideTitle\par}
  \vspace{0.25in}
  {\Large \ProgramName\par}
  \vspace{0.35in}
  {\large \DocVersion\par}
  \vspace{0.10in}
  {\large \DocDate\par}
  \vfill
  \begin{minipage}{0.93\textwidth}
  \small
  \textbf{Purpose.} Define a rigorous, layperson-accessible program charter for an open, reproducible neuromorphic hardware effort. This guide locks scope, success metrics, benchmarks, measurement boundaries, phase gates, and pivot rules so the program converges.
  \\
  \\
  \textbf{Safety note.} This guide is an engineering planning document. It is not a manual for constructing hazardous systems (high-voltage electron columns, vacuum pressure vessels, toxic gases). It assumes fabrication is performed in professional/shared facilities or with commercial systems, under appropriate safety controls and trained operators.
  \\
  \\
  \textbf{Materials note.} The program assumes silver (Ag) is used where technically advantageous. The guide treats Ag as operationally restricted in contamination-control terms (tool zoning, sample states, late introduction discipline).
  \end{minipage}
  \vspace*{0.6in}
\end{titlepage}

\tableofcontents
\newpage

\section{At a Glance}

\subsection{Who this guide is for}
This guide assumes the operator:
\begin{itemize}[leftmargin=*,itemsep=2pt]
\item has strong electronics technician lab skills (instrumentation, debugging, PCB work, ESD discipline),
\item has adequate funding and access to advanced software assistance,
\item is comfortable with bachelor-level physics/circuits/materials concepts,
\item does \emph{not} necessarily have deep cleanroom process development experience.
\end{itemize}

\subsection{What this guide produces (deliverables)}
By completing this guide, you will produce five core documents that govern the entire project:

\begin{enumerate}[leftmargin=*,itemsep=2pt]
\item \textbf{Program Charter (1--3 pages).} Mission, scope, non-goals, guiding principles, definition of success.
\item \textbf{KPI \& Measurement Boundary Sheet (1--2 pages).} Exactly what metrics you report and what is included/excluded.
\item \textbf{Benchmark Plan (1--2 pages).} The workloads used to justify ``processor-equivalent'' outcomes.
\item \textbf{Phase Gates \& Stop/Pivot Rules (1--2 pages).} When you advance; when you iterate; when you change direction.
\item \textbf{Decision Log + Risk Register (living documents).} Traceable choices, risks, mitigations, triggers, and owners.
\end{enumerate}

\subsection{Why this matters}
Hardware R\&D programs fail most often due to:
\begin{itemize}[leftmargin=*,itemsep=2pt]
\item unbounded scope (``we can also add X, Y, Z''),
\item metrics that are not comparable run-to-run,
\item energy/performance claims that shift due to unclear measurement boundaries,
\item lack of pivot rules (teams keep iterating on a dead-end stack),
\item poor reproducibility (no raw data, missing metadata, manual plots).
\end{itemize}
This guide is designed to eliminate those failure modes early.

\section{Before You Start: A Mental Model That Prevents Scope Collapse}

\subsection{The single most important framing decision}
Do not anchor the program on ``build a modern laptop CPU from scratch.'' That is a manufacturing-scale objective requiring leading-edge lithography, extreme overlay control, high-yield FEOL transistor formation, and industrial statistical process control.

Instead, define success in one of these credible meanings:
\begin{enumerate}[leftmargin=*,itemsep=2pt]
\item \textbf{Workload-equivalent compute (recommended).} ``Equivalent'' means: on a chosen workload, your tile hits a target for energy, accuracy, and (optionally) latency.
\item \textbf{Hybrid module equivalence (also recommended).} ``Equivalent'' means: a packaged module containing your tile plus a commodity controller can replace a CPU for a specific embedded job.
\item \textbf{General-purpose ISA CPU equivalence (not recommended early).} Treat this as long-term aspiration, not a phase gate.
\end{enumerate}

\subsection{A simple rule that prevents scope creep}
\textbf{Rule:} Every new idea must tie to one of your defined KPIs and must have a measurement plan. If you cannot measure it, you do not build it.

\section{Critical Cliffs and Program Controls (Non-Optional)}
These are the points where projects most commonly die. They are treated as explicit constraints and gates:
\begin{itemize}[leftmargin=*,itemsep=2pt]
\item \textbf{Mobile-metal operational segregation (Ag).} Plan toolchain zones, sample states, and late introduction. Encapsulation reduces risk; it does not eliminate risk.
\item \textbf{Active dielectric quality.} Do not interpret leakage-dominated behavior as neuromorphic dynamics. Active dielectric quality gates are mandatory.
\item \textbf{Passive-array scaling wall.} Passive arrays are characterization-first unless a selector strategy or transistor backplane is proven under the intended bias scheme.
\item \textbf{Scoring throughput.} Scoring must not become per-device. Scoring budgets are KPIs and gates.
\item \textbf{Packaging survivability.} Pad/probing/packaging constraints are design-time requirements, not late-stage integration tasks.
\end{itemize}

\section{Step-by-Step Procedure}

\subsection{Step 1: Write a one-paragraph mission}
\textbf{Goal:} Produce a single paragraph a facility manager, PI, or collaborator can understand immediately.

Write the mission using this structure:
\begin{enumerate}[leftmargin=*,itemsep=2pt]
\item \textbf{Problem:} what limitation you are targeting (energy, latency, always-on sensing, etc.).
\item \textbf{Hypothesis:} why volatile/diffusive memristive dynamics can help (physics-native temporal processing).
\item \textbf{Deliverable:} what you will deliver physically and digitally (devices, arrays, tiles, open datasets, models).
\item \textbf{Impact:} why it matters.
\end{enumerate}

\subsection{Step 2: Define scope, non-goals, and guiding principles}
\textbf{Goal:} Bound the program so it can finish.

\subsubsection{2.1 In scope (recommended)}
\begin{itemize}[leftmargin=*,itemsep=2pt]
\item Single-device test structures for volatile switching characterization.
\item Neuron-like spiking primitives (external circuitry acceptable early).
\item Small arrays (e.g., $16\times16$ to $128\times128$) and yield mapping.
\item Hybrid integration approaches (BEOL memristor layers on pre-fabricated CMOS, where feasible).
\item Open measurement automation, data schemas, compact models, and benchmark harnesses.
\end{itemize}

\subsubsection{2.2 Out of scope (recommended)}
\begin{itemize}[leftmargin=*,itemsep=2pt]
\item Building a modern laptop-class CPU from raw silicon in-house.
\item Any attempt to obtain proprietary trade secrets or non-public process details.
\item Instructions for constructing hazardous equipment (HV electron columns, vacuum pressure vessels, toxic gas systems).
\end{itemize}

\subsubsection{2.3 Guiding principles (choose 4--8)}
Recommended principles:
\begin{itemize}[leftmargin=*,itemsep=2pt]
\item \textbf{No data, no claim.}
\item \textbf{Every run produces a dataset.}
\item \textbf{Scoring is a scalpel, not a factory.}
\item \textbf{Maskless lithography is the backbone.}
\item \textbf{Reproducibility over hero demos.}
\item \textbf{Document deviations.}
\end{itemize}

\subsection{Step 3: Define what ``processor-equivalent'' means}
\textbf{Goal:} Lock a measurable equivalence claim that you can win.

Choose one primary claim type:
\begin{itemize}[leftmargin=*,itemsep=2pt]
\item \textbf{Workload-equivalent}: win on a specific temporal workload class.
\item \textbf{Hybrid module}: win as a tile + controller module on a specific embedded job.
\end{itemize}

\textbf{Rule:} In early releases, compare on at most \textbf{two} primary axes (e.g., energy per task and accuracy).

\subsection{Step 4: Define phases and gates}
\textbf{Goal:} Prevent premature scaling and enforce sequence.

\subsubsection{Phase P0: Single-device truth}
\textbf{Gate to exit P0 must include:}
\begin{itemize}[leftmargin=*,itemsep=2pt]
\item evidence that behavior is not dominated by dielectric leakage or measurement artifacts,
\item repeatable switching regime across a meaningful number of devices,
\item at least one compact model fit that reproduces key measured behaviors.
\end{itemize}

\subsubsection{Phase P1: Neuron primitive truth}
\textbf{Gate to exit P1 must include:}
\begin{itemize}[leftmargin=*,itemsep=2pt]
\item repeatable spiking-like behavior under scripted stimulation,
\item documented parameter sweeps stable across multiple devices and at least two runs.
\end{itemize}

\subsubsection{Phase P2: Array truth}
\textbf{Gate to exit P2 must include:}
\begin{itemize}[leftmargin=*,itemsep=2pt]
\item array yield map and failure classification with evidence,
\item explicit statement whether array is characterization-only or addressable compute,
\item if claiming addressable scaling: demonstrated selector strategy or transistor backplane viability under intended bias scheme.
\end{itemize}

\subsubsection{Phase P3: Tile/system truth}
\textbf{Gate to exit P3 must include:}
\begin{itemize}[leftmargin=*,itemsep=2pt]
\item packageable die with validated pads and harness,
\item end-to-end benchmark run with honest measurement boundaries.
\end{itemize}

\subsection{Step 5: Select 1--2 benchmark workloads and define win conditions}
For each benchmark define:
\begin{itemize}[leftmargin=*,itemsep=2pt]
\item required accuracy $A_{\min}$,
\item maximum energy per task $E_{\mathrm{task,max}}$,
\item optional latency bound $T_{\max}$,
\item dataset and preprocessing version locks.
\end{itemize}

\subsection{Step 6: Define KPI sets and measurement boundaries}
You need KPIs at device, neuron, array, and tile levels, plus explicit measurement boundaries.

\subsubsection{Pulse energy (continuous-time definition)}
\[
E \;=\; \int_{t_0}^{t_1} V(t)\, I(t)\, dt
\]

\textbf{Explainer table (every symbol defined):}
\begin{center}
\begin{tabular}{@{}P{2.2cm}P{9.7cm}P{2.0cm}@{}}
\toprule
\textbf{Symbol} & \textbf{Meaning} & \textbf{Units} \\
\midrule
$E$ & Electrical energy consumed during the event window & joule \\
$t$ & Time (continuous variable) & second \\
$t_0$ & Start time of the integration window & second \\
$t_1$ & End time of the integration window & second \\
$V(t)$ & Voltage across the device as a function of time & volt \\
$I(t)$ & Current through the device as a function of time & ampere \\
$\int$ & Integration operator over time & none \\
\bottomrule
\end{tabular}
\end{center}

\subsubsection{Pulse energy (sampled-data approximation)}
\[
E \;\approx\; \sum_{k=0}^{N-1} V_k\, I_k\, \Delta t
\]

\textbf{Explainer table (every symbol defined):}
\begin{center}
\begin{tabular}{@{}P{2.2cm}P{9.7cm}P{2.0cm}@{}}
\toprule
\textbf{Symbol} & \textbf{Meaning} & \textbf{Units} \\
\midrule
$E$ & Approximate electrical energy over the sampled window & joule \\
$k$ & Sample index & none \\
$N$ & Number of samples & none \\
$V_k$ & Voltage sample at index $k$ & volt \\
$I_k$ & Current sample at index $k$ & ampere \\
$\Delta t$ & Sampling interval & second \\
$\sum$ & Summation over sample index & none \\
\bottomrule
\end{tabular}
\end{center}

\subsection{Step 7: Define stop/pivot rules (tiered pivots)}
\textbf{Goal:} Avoid infinite iteration on a dead-end path.

\subsubsection{Tiered pivot framework}
\begin{enumerate}[leftmargin=*,itemsep=2pt]
\item \textbf{Pivot Tier 1 (within neuromorphic):} pivot architecture/circuits first.
\item \textbf{Pivot Tier 2 (integration pivot):} pivot to 1T1R earlier if sneak paths block scaling.
\item \textbf{Pivot Tier 3 (deliverable pivot):} pivot deliverable class if trend and root-cause do not support continuation.
\end{enumerate}

\subsection{Step 8: Define open reproducibility requirements}
Every published claim must include raw data, metadata, scripts, and a README.

\subsection{Step 9: Build the initial risk register}
Include the cliffs explicitly: Ag segregation, dielectric leakage, lift-off fences, sneak paths, scoring throughput, packaging yield, measurement artifacts, data management.

\subsection{Step 10: Assemble the Program Charter packet}
Bundle: charter, equivalence definition, phases/gates, benchmark plan, KPI/boundary sheet, pivots, and risk register.

\appendix
\section{Appendix A: Templates}

\subsection{A1. Program Charter template (headings)}
\begin{enumerate}[leftmargin=*,itemsep=2pt]
\item Mission
\item Why now
\item Scope and non-goals
\item Definition of success
\item Phases and gates
\item Benchmarks
\item KPIs
\item Measurement boundaries
\item Stop/pivot rules
\item Reproducibility policy
\item Risks and mitigations
\end{enumerate}

\subsection{A2. KPI sheet template (headings)}
\begin{itemize}[leftmargin=*,itemsep=2pt]
\item Device KPIs
\item Neuron KPIs
\item Array KPIs
\item Tile/system KPIs
\item Notes and caveats
\end{itemize}

\subsection{A3. Decision log template (fields)}
\begin{itemize}[leftmargin=*,itemsep=2pt]
\item Date
\item Decision
\item Options considered
\item Evidence used
\item Risk assessment
\item Next actions
\item Owner
\end{itemize}

\subsection{A4. Risk register template (fields)}
\begin{itemize}[leftmargin=*,itemsep=2pt]
\item Risk ID
\item Description
\item Severity
\item Likelihood
\item Trigger
\item Mitigation
\item Owner
\item Status
\end{itemize}

\section{Appendix B: ``Max Capability'' Leverage (AI + Funding)}
\subsection{B1. Auditable software assistance}
\begin{itemize}[leftmargin=*,itemsep=2pt]
\item auto-generate travelers and checklists from a single source of truth,
\item auto-check dataset completeness,
\item auto-generate run summaries with human sign-off,
\item propose DOE parameter sweeps to reduce trial count.
\end{itemize}

\subsection{B2. Spend on bottlenecks with compounding returns}
\begin{itemize}[leftmargin=*,itemsep=2pt]
\item measurement automation,
\item ALD access for active dielectrics,
\item planarity access,
\item packaging and repeatable test harnesses.
\end{itemize}

\end{document}
