% Standalone LaTeX file (Overleaf-ready)
% Guide 10: Single-Device Characterization Suite
% Compile with: PDFLaTeX

\documentclass[11pt]{article}

\usepackage[margin=1in]{geometry}
\usepackage{microtype}
\usepackage{xcolor}
\usepackage{hyperref}
\usepackage{booktabs}
\usepackage{longtable}
\usepackage{array}
\usepackage{enumitem}
\usepackage{siunitx}
\usepackage{fancyhdr}

\definecolor{darkblue}{rgb}{0.0, 0.0, 0.5}
\hypersetup{
  colorlinks=true,
  linkcolor=darkblue,
  urlcolor=darkblue,
  citecolor=darkblue,
  pdftitle={Guide 10: Single-Device Characterization Suite},
  pdfauthor={Open Research Draft}
}

\sisetup{detect-all=true}

\newcommand{\ProgramName}{Layer--Score--Layer Neuromorphic Tile Program}
\newcommand{\GuideID}{Guide 10}
\newcommand{\GuideTitle}{Single-Device Characterization Suite}
\newcommand{\DocVersion}{v1.1 (standalone)}
\newcommand{\DocDate}{\today}

\setlength{\parindent}{0pt}
\setlength{\parskip}{0.55em}

\pagestyle{fancy}
\fancyhf{}
\lhead{\GuideID: \GuideTitle}
\rhead{\DocVersion}
\cfoot{\thepage}

\newcolumntype{P}[1]{>{\raggedright\arraybackslash}p{#1}}

\begin{document}

\begin{titlepage}
  \centering
  \vspace*{1.0in}
  {\LARGE \GuideID: \GuideTitle\par}
  \vspace{0.25in}
  {\Large \ProgramName\par}
  \vspace{0.35in}
  {\large \DocVersion\par}
  \vspace{0.10in}
  {\large \DocDate\par}
  \vfill
  \begin{minipage}{0.93\textwidth}
  \small
  \textbf{Purpose.} Define a standardized electrical and metrology characterization suite for single devices that produces publishable switching statistics, includes artifact controls, and outputs reproducible datasets suitable for modeling and benchmarking.
  \\
  \\
  \textbf{Safety note.} This guide assumes use of professional instrumentation and facility-approved probing/handling practices.
  \end{minipage}
  \vspace*{0.6in}
\end{titlepage}

\tableofcontents
\newpage

\section{Core Deliverables}
\begin{enumerate}[leftmargin=*,itemsep=2pt]
\item Instrumentation configuration standard (SMUs, pulse generator, oscilloscope, probe station) and calibration procedure.
\item Standard test definitions: DC sweeps (where appropriate), pulsed threshold tests, volatility decay, endurance cycling, optional temperature dependence.
\item Data schema (raw waveforms + metadata + derived metrics) and naming conventions.
\item Failure classification labels (open, short, stuck-on, stuck-off, drift) and annotation rules.
\item Minimum sample size requirements for any claim (number of devices and number of cycles).
\item Artifact check protocol (controls to rule out test-setup-induced behavior).
\end{enumerate}

\section{Why This Guide Exists (Layperson Summary)}
In early memristor work, it is easy to ``see'' spikes, switching, or analog behavior that is actually caused by:
\begin{itemize}[leftmargin=*,itemsep=2pt]
\item measurement bandwidth limitations,
\item cabling and grounding artifacts,
\item compliance current settings,
\item leakage through poor dielectrics,
\item probe contact instability.
\end{itemize}
This guide forces a disciplined measurement workflow so that device claims are credible and comparable across runs.

\section{Step-by-Step Procedure}

\subsection{Step 1: Define and record the instrumentation stack}
\begin{enumerate}[leftmargin=*,itemsep=2pt]
\item List instruments and identifiers (model, serial number) used for each measurement role.
\item Define cabling and probe configuration (including any amplifiers or filters).
\item Record calibration state and date.
\item Record measurement bandwidth (scope/sample rate) for waveform captures.
\end{enumerate}

\subsection{Step 2: Establish a standard dataset structure}
\begin{enumerate}[leftmargin=*,itemsep=2pt]
\item Define run IDs that link back to the fabrication traveler.
\item Store raw waveform data when feasible (do not only store extracted features).
\item Store metadata: device coordinates, pad ID, toolchain state, temperature (if measured), and instrument settings.
\end{enumerate}

\subsection{Step 3: Execute the standard test suite}
\textbf{Minimum recommended suite:}
\begin{itemize}[leftmargin=*,itemsep=2pt]
\item \textbf{Threshold characterization (pulsed):} determine the distribution of switching thresholds under defined pulse widths.
\item \textbf{Volatility/decay:} measure relaxation back to baseline conductance as a function of time.
\item \textbf{Endurance/drift:} cycle devices and track threshold/decay drift.
\item \textbf{Optional:} temperature dependence and hold-time dependence.
\end{itemize}

\subsection{Step 4: Compute and report distributions}
\begin{enumerate}[leftmargin=*,itemsep=2pt]
\item Compute device-to-device and cycle-to-cycle distributions (threshold, decay constant, event energy).
\item Report summary statistics and distribution plots using scripts.
\item Store derived metrics alongside raw data with clear provenance.
\end{enumerate}

\subsection{Step 5: Run artifact checks (non-negotiable)}
\textbf{Controls should include:}
\begin{itemize}[leftmargin=*,itemsep=2pt]
\item open-circuit or dummy structures (no active device) under the same stimulation,
\item bandwidth sanity checks (repeat with different sampling rates where practical),
\item probe contact stability checks,
\item compliance-current and series-resistance sensitivity checks.
\end{itemize}

\section{Minimum Sample Sizes (Policy)}
Any claim intended for external release should, at minimum, specify:
\begin{itemize}[leftmargin=*,itemsep=2pt]
\item number of devices measured $N_{\mathrm{dev}}$,
\item number of cycles per device $N_{\mathrm{cyc}}$,
\item number of independent fabrication runs $N_{\mathrm{run}}$.
\end{itemize}
(If you formalize these symbols in equations, include a symbol-definition table.)

\section{Exit Criteria}
Guide 10 is complete when:
\begin{itemize}[leftmargin=*,itemsep=2pt]
\item two independent runs produce comparable datasets using the same schema and scripts,
\item raw waveforms and metadata are sufficient to regenerate plots end-to-end,
\item artifact checks demonstrate the behavior is not measurement-induced,
\item failure modes are classified consistently.
\end{itemize}

\end{document}
