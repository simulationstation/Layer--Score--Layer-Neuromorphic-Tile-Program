% Standalone LaTeX file (Overleaf-ready)
% Guide 13: Compact Model Calibration and Hardware-in-the-Loop Simulation
% Compile with: PDFLaTeX

\documentclass[11pt]{article}

\usepackage[margin=1in]{geometry}
\usepackage{microtype}
\usepackage{xcolor}
\usepackage{hyperref}
\usepackage{booktabs}
\usepackage{longtable}
\usepackage{array}
\usepackage{enumitem}
\usepackage{siunitx}
\usepackage{fancyhdr}

\definecolor{darkblue}{rgb}{0.0, 0.0, 0.5}
\hypersetup{
  colorlinks=true,
  linkcolor=darkblue,
  urlcolor=darkblue,
  citecolor=darkblue,
  pdftitle={Guide 13: Compact Model Calibration and Hardware-in-the-Loop Simulation},
  pdfauthor={Open Research Draft}
}

\sisetup{detect-all=true}

\newcommand{\ProgramName}{Layer--Score--Layer Neuromorphic Tile Program}
\newcommand{\GuideID}{Guide 13}
\newcommand{\GuideTitle}{Compact Model Calibration and Hardware-in-the-Loop Simulation}
\newcommand{\DocVersion}{v1.1 (standalone)}
\newcommand{\DocDate}{\today}

\setlength{\parindent}{0pt}
\setlength{\parskip}{0.55em}

\pagestyle{fancy}
\fancyhf{}
\lhead{\GuideID: \GuideTitle}
\rhead{\DocVersion}
\cfoot{\thepage}

\newcolumntype{P}[1]{>{\raggedright\arraybackslash}p{#1}}

\begin{document}

\begin{titlepage}
  \centering
  \vspace*{1.0in}
  {\LARGE \GuideID: \GuideTitle\par}
  \vspace{0.25in}
  {\Large \ProgramName\par}
  \vspace{0.35in}
  {\large \DocVersion\par}
  \vspace{0.10in}
  {\large \DocDate\par}
  \vfill
  \begin{minipage}{0.93\textwidth}
  \small
  \textbf{Purpose.} Convert measured device behavior into predictive compact models that support design iteration, algorithm co-design, and benchmark evaluation under realistic variability.
  \\
  \\
  \textbf{Scope note.} This guide defines modeling and validation workflows; it does not prescribe proprietary device physics equations. Models should be selected based on measurability, calibratability, and predictive utility.
  \end{minipage}
  \vspace*{0.6in}
\end{titlepage}

\tableofcontents
\newpage

\section{Core Deliverables}
\begin{enumerate}[leftmargin=*,itemsep=2pt]
\item Model family selection (fast, calibratable, aligned to measured dynamics)
\item Parameter extraction scripts tied to measured datasets
\item Validation protocol (withheld datasets; error reporting)
\item Hardware-in-the-loop benchmark harness (simulation uses measured distributions)
\item Model versioning and compatibility policy
\end{enumerate}

\section{Step-by-Step Procedure}
\begin{enumerate}[leftmargin=*,itemsep=2pt]
\item \textbf{Select a minimal model family.} Choose a model that reproduces key measured observables (threshold behavior, volatility/decay, stochasticity distributions) while remaining simple enough to fit automatically.
\item \textbf{Define parameter extraction mapping.} Specify which measured curves determine which parameters; record this mapping in code and documentation.
\item \textbf{Automate fitting.} Implement scripts that produce parameter sets with confidence indicators (fit residuals, parameter bounds) and store them as versioned artifacts.
\item \textbf{Validate on withheld data.} Hold out devices and entire runs. Report errors and identify failure regimes.
\item \textbf{Integrate into a benchmark harness.} Use measured variability distributions (device-to-device and cycle-to-cycle) in simulations and training loops.
\item \textbf{Publish model artifacts.} Release model code, parameter sets, and validation scripts as part of each tagged program release (Guide 15).
\end{enumerate}

\section{Common Failure Modes (and mitigations)}
\begin{itemize}[leftmargin=*,itemsep=2pt]
\item \textbf{Overfitting to one run.} Mitigate by withholding entire runs and validating cross-run.
\item \textbf{Model too slow.} Mitigate by simplifying and using piecewise approximations where justified.
\item \textbf{Model not tied to measurable quantities.} Mitigate by restricting to parameters that map directly to measured observables.
\end{itemize}

\section{Exit Criteria}
Guide 13 is complete when:
\begin{itemize}[leftmargin=*,itemsep=2pt]
\item model predictions are validated on withheld devices/runs,
\item the model is predictive enough to guide a design decision,
\item a subsequent fabrication run validates that decision directionally,
\item model artifacts are reproducible from tagged releases.
\end{itemize}

\end{document}
