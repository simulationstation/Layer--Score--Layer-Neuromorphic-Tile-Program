% Standalone LaTeX file (Overleaf-ready)
% Guide 11: Neuron Behavior Bring-up
% Compile with: PDFLaTeX

\documentclass[11pt]{article}

\usepackage[margin=1in]{geometry}
\usepackage{microtype}
\usepackage{xcolor}
\usepackage{hyperref}
\usepackage{booktabs}
\usepackage{longtable}
\usepackage{array}
\usepackage{enumitem}
\usepackage{siunitx}
\usepackage{fancyhdr}

\definecolor{darkblue}{rgb}{0.0, 0.0, 0.5}
\hypersetup{
  colorlinks=true,
  linkcolor=darkblue,
  urlcolor=darkblue,
  citecolor=darkblue,
  pdftitle={Guide 11: Neuron Behavior Bring-up},
  pdfauthor={Open Research Draft}
}

\sisetup{detect-all=true}

\newcommand{\ProgramName}{Layer--Score--Layer Neuromorphic Tile Program}
\newcommand{\GuideID}{Guide 11}
\newcommand{\GuideTitle}{Neuron Behavior Bring-up}
\newcommand{\DocVersion}{v1.1 (standalone)}
\newcommand{\DocDate}{\today}

\setlength{\parindent}{0pt}
\setlength{\parskip}{0.55em}

\pagestyle{fancy}
\fancyhf{}
\lhead{\GuideID: \GuideTitle}
\rhead{\DocVersion}
\cfoot{\thepage}

\begin{document}

\begin{titlepage}
  \centering
  \vspace*{1.0in}
  {\LARGE \GuideID: \GuideTitle\par}
  \vspace{0.25in}
  {\Large \ProgramName\par}
  \vspace{0.35in}
  {\large \DocVersion\par}
  \vspace{0.10in}
  {\large \DocDate\par}
  \vfill
  \begin{minipage}{0.93\textwidth}
  \small
  \textbf{Purpose.} Prove neuron-like spiking behavior in a controlled, measurable, repeatable way, and define promotion criteria from single-device physics (Guide 10) to arrays (Guide 12).
  \\
  \\
  \textbf{Safety note.} This guide assumes professional instrumentation and safe operating practice. It is not a manual for hazardous equipment construction.
  \end{minipage}
  \vspace*{0.6in}
\end{titlepage}

\tableofcontents
\newpage

\section{Core Deliverables}
\begin{enumerate}[leftmargin=*,itemsep=2pt]
\item Minimal neuron test harness definition (external circuitry acceptable initially)
\item Standard stimulation protocols (pulse trains) and expected observables
\item Parameter sweep plan (biases, pulse width, spacing) with logging requirements
\item Metrics and promotion criteria for advancing to arrays
\end{enumerate}

\section{Layperson Summary}
A ``neuron primitive'' here means a hardware element that produces spike-like events (sudden conductance changes or voltage/current pulses) in response to inputs and then resets (volatility) without complex digital control. Early stages may rely on external resistors/transistors/controllers to create a stable test harness. The objective is not to claim a biological neuron is replicated perfectly; it is to demonstrate repeatable, tunable spiking dynamics that are useful for temporal processing.

\section{Step-by-Step Procedure}

\subsection{Step 1: Define the minimal test harness}
\textbf{Goal:} Create a controlled environment for stimulating the device and measuring response.

\begin{enumerate}[leftmargin=*,itemsep=2pt]
\item Define input stimulus source (pulse generator or SMU pulse mode) and output measurement path (scope/SMU) with bandwidth requirements.
\item Define any external circuit elements used (e.g., resistor network, compliance elements, transistor gating).
\item Define measurement reference points (where voltage is measured, where current is sensed).
\item Calibrate the setup and record calibration state in the dataset metadata.
\end{enumerate}

\subsection{Step 2: Define standard stimulation protocols}
\textbf{Goal:} Use consistent inputs so results are comparable across devices and runs.

\begin{itemize}[leftmargin=*,itemsep=2pt]
\item Single-pulse threshold probe protocol
\item Pulse-train protocol for spiking-rate response (frequency sweep)
\item Burst protocol to evaluate adaptation/history dependence
\item Idle/rest protocol to quantify reset/volatility behavior
\end{itemize}

\subsection{Step 3: Execute parameter sweeps}
\textbf{Goal:} Establish tunability and stability.

\begin{enumerate}[leftmargin=*,itemsep=2pt]
\item Sweep pulse amplitude while holding width/spacing fixed.
\item Sweep pulse width while holding amplitude/spacing fixed.
\item Sweep inter-pulse interval (spacing) while holding amplitude/width fixed.
\item For each sweep, store raw waveforms when feasible and store derived metrics with full metadata.
\end{enumerate}

\subsection{Step 4: Compute neuron-level metrics}
\textbf{Minimum metrics:}
\begin{itemize}[leftmargin=*,itemsep=2pt]
\item Spike rate as a function of input stimulus
\item Refractory-like behavior (if present): minimum time between spikes under fixed input
\item Adaptation (if claimed): spike-rate change over repeated stimulation
\item Stochasticity: distribution of spike timing or threshold under repeated identical stimuli
\end{itemize}

\subsection{Step 5: Apply promotion criteria (P1 gate)}
\textbf{Promotion criteria examples:}
\begin{itemize}[leftmargin=*,itemsep=2pt]
\item Behavior is reproducible across multiple devices, not a single ``hero'' device.
\item Behavior persists across at least two fabrication runs (process repeatability).
\item Artifact checks indicate the setup is not generating the spikes (e.g., grounding, bandwidth, compliance artifacts).
\item The behavior is tunable within a useful range (for the chosen benchmark class).
\end{itemize}

\section{Exit Criteria}
Guide 11 is complete when neuron-like behavior is demonstrated reproducibly across multiple devices and at least two fabrication runs, with scripted stimulation and data logging sufficient for an external replicator to rerun the analysis.

\end{document}
