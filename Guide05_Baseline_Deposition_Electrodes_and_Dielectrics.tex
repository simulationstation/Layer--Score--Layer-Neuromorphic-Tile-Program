% Standalone LaTeX file (Overleaf-ready)
% Guide 05: Baseline Deposition: Electrodes and Dielectrics
% Compile with: PDFLaTeX

\documentclass[11pt]{article}

\usepackage[margin=1in]{geometry}
\usepackage{microtype}
\usepackage{xcolor}
\usepackage{hyperref}
\usepackage{booktabs}
\usepackage{longtable}
\usepackage{array}
\usepackage{enumitem}
\usepackage{siunitx}
\usepackage{fancyhdr}

\definecolor{darkblue}{rgb}{0.0, 0.0, 0.5}
\hypersetup{
  colorlinks=true,
  linkcolor=darkblue,
  urlcolor=darkblue,
  citecolor=darkblue,
  pdftitle={Guide 05: Baseline Deposition: Electrodes and Dielectrics},
  pdfauthor={Open Research Draft}
}

\sisetup{detect-all=true}

\newcommand{\ProgramName}{Layer--Score--Layer Neuromorphic Tile Program}
\newcommand{\GuideID}{Guide 05}
\newcommand{\GuideTitle}{Baseline Deposition: Electrodes and Dielectrics}
\newcommand{\DocVersion}{v1.1 (standalone)}
\newcommand{\DocDate}{\today}

\setlength{\parindent}{0pt}
\setlength{\parskip}{0.55em}

\pagestyle{fancy}
\fancyhf{}
\lhead{\GuideID: \GuideTitle}
\rhead{\DocVersion}
\cfoot{\thepage}

\begin{document}

\begin{titlepage}
  \centering
  \vspace*{1.0in}
  {\LARGE \GuideID: \GuideTitle\par}
  \vspace{0.25in}
  {\Large \ProgramName\par}
  \vspace{0.35in}
  {\large \DocVersion\par}
  \vspace{0.10in}
  {\large \DocDate\par}
  \vfill
  \begin{minipage}{0.93\textwidth}
  \small
  \textbf{Purpose.} Standardize thin-film stacks and verification gates so device behavior is interpretable and repeatable. This guide explicitly prevents dielectric leakage and topography defects from masquerading as ``interesting'' switching physics.
  \\
  \\
  \textbf{Key policy update.} The \emph{active switching dielectric} is \textbf{ALD-by-default}. Sputtered dielectrics may be used for the active layer only with documented rationale and demonstrated repeatability across runs.
  \end{minipage}
  \vspace*{0.6in}
\end{titlepage}

\tableofcontents
\newpage

\section{Core Deliverables}
\begin{enumerate}[leftmargin=*,itemsep=2pt]
\item Baseline electrode stacks (including Ag introduction plan)
\item Baseline dielectric recipes (with active-layer method declared)
\item Film verification checklist (thickness, roughness proxy, sheet resistance where relevant)
\item Film quality gates and rework policy
\item Leakage sanity-check protocol (before switching claims)
\item Lot tracking for targets, precursors, and key consumables
\end{enumerate}

\section{Active Switching Dielectric Policy (ALD Default)}
\textbf{Policy:} For the \emph{active switching dielectric layer}, \textbf{ALD is the default}.

Sputtered dielectrics may be used for the active layer only if:
\begin{itemize}[leftmargin=*,itemsep=2pt]
\item you document a rationale (e.g., proven equivalence in your data),
\item you pass leakage sanity checks and film quality gates, and
\item you show run-to-run reproducibility.
\end{itemize}

Sputtered dielectrics remain acceptable for \textbf{passivation/encapsulation} layers where the primary function is coverage and protection rather than controlled switching physics.

\section{Step-by-Step Procedure (High-Level, Facility-Compliant)}
\begin{enumerate}[leftmargin=*,itemsep=2pt]
\item Choose a baseline stack and freeze it for the next two to three runs (avoid changing everything at once).
\item For each film, define mandatory recorded parameters: tool ID, recipe ID, temperature, time, pre-clean used, and target/precursor lot.
\item Define verification per film:
  \begin{itemize}[leftmargin=*,itemsep=2pt]
  \item thickness measurement method and standard measurement locations,
  \item surface inspection sites,
  \item sheet resistance measurement method (if relevant).
  \end{itemize}
\item Define ``film quality gates'' (pass/fail) and actions if a film fails (rework versus scrap).
\item Define the step at which Ag is introduced and prefer \textbf{late introduction} where feasible.
\item After Ag introduction, explicitly list allowed/forbidden tool zones and steps (Guide 02 toolchain segregation).
\end{enumerate}

\section{Leakage Sanity-Check Gate (Minimum)}
Before interpreting any behavior as switching dynamics, require:
\begin{itemize}[leftmargin=*,itemsep=2pt]
\item a pre-switch leakage measurement on representative test structures,
\item a documented leakage threshold beyond which the film is considered unsuitable for physics claims,
\item a recorded decision: ``film accepted'' or ``film rejected'' with evidence.
\end{itemize}

\section{Film Verification Checklist (Suggested)}
\begin{itemize}[leftmargin=*,itemsep=2pt]
\item Thickness: measured at standard sites; record mean and range.
\item Surface inspection: optical images at standard sites; note particulate and defects.
\item Roughness proxy: profilometry/AFM where available; record values and method.
\item Sheet resistance (metals): record and compare to expected range.
\item Adhesion sanity check: qualitative note (peel, lift, cracking observed or not).
\end{itemize}

\section{Exit Criteria}
Guide 05 is complete when you can produce two independent runs in which:
\begin{itemize}[leftmargin=*,itemsep=2pt]
\item baseline films meet thickness/quality gates,
\item leakage sanity checks pass for the active dielectric,
\item device behavior remains in the same qualitative regime,
\item differences are attributable to documented changes rather than undocumented deposition drift.
\end{itemize}

\end{document}
