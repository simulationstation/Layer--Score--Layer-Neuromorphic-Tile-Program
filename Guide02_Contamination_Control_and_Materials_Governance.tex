% Standalone LaTeX file (Overleaf-ready)
% Guide 02: Contamination Control and Materials Governance
% Compile with: PDFLaTeX

\documentclass[11pt]{article}

\usepackage[margin=1in]{geometry}
\usepackage{microtype}
\usepackage{xcolor}
\usepackage{hyperref}
\usepackage{booktabs}
\usepackage{longtable}
\usepackage{array}
\usepackage{enumitem}
\usepackage{siunitx}
\usepackage{fancyhdr}

\definecolor{darkblue}{rgb}{0.0, 0.0, 0.5}
\hypersetup{
  colorlinks=true,
  linkcolor=darkblue,
  urlcolor=darkblue,
  citecolor=darkblue,
  pdftitle={Guide 02: Contamination Control and Materials Governance},
  pdfauthor={Open Research Draft}
}

\sisetup{detect-all=true}

\newcommand{\ProgramName}{Layer--Score--Layer Neuromorphic Tile Program}
\newcommand{\GuideID}{Guide 02}
\newcommand{\GuideTitle}{Contamination Control and Materials Governance}
\newcommand{\DocVersion}{v1.1 (standalone)}
\newcommand{\DocDate}{\today}

\setlength{\parindent}{0pt}
\setlength{\parskip}{0.55em}

\pagestyle{fancy}
\fancyhf{}
\lhead{\GuideID: \GuideTitle}
\rhead{\DocVersion}
\cfoot{\thepage}

\newcolumntype{P}[1]{>{\raggedright\arraybackslash}p{#1}}

\begin{document}

\begin{titlepage}
  \centering
  \vspace*{1.0in}
  {\LARGE \GuideID: \GuideTitle\par}
  \vspace{0.25in}
  {\Large \ProgramName\par}
  \vspace{0.35in}
  {\large \DocVersion\par}
  \vspace{0.10in}
  {\large \DocDate\par}
  \vfill
  \begin{minipage}{0.93\textwidth}
  \small
  \textbf{Purpose.} Establish facility-compliant (or internally compliant) handling of restricted materials (especially mobile metals such as Ag) to prevent contamination events, protect yield, and ensure samples can move between tools/workflows safely.
  \\
  \\
  \textbf{Key principle.} Encapsulation reduces risk; it does not eliminate risk. This guide therefore requires explicit sample-state labeling and explicit toolchain segregation (Green/Yellow/Red zones) whenever mobile metals are involved.
  \end{minipage}
  \vspace*{0.6in}
\end{titlepage}

\tableofcontents
\newpage

\section{Scope and Audience}
\subsection{Audience and assumptions}
This guide assumes you have strong electronics lab discipline (ESD, labeling, careful handling) but may be unfamiliar with:
\begin{itemize}[leftmargin=*,itemsep=2pt]
\item cleanroom contamination classes and toolset restrictions,
\item why mobile metals require toolchain segregation,
\item why encapsulation is not a universal ``safe passport'' into shared tools.
\end{itemize}

\subsection{Core deliverables}
\begin{enumerate}[leftmargin=*,itemsep=2pt]
\item Materials registry (elements/materials present, including trace dopants and residues)
\item Sample state classification standard (S0--S3)
\item Toolchain segregation plan (Green/Yellow/Red zones)
\item Tool compatibility matrix (which sample states can go to which tool classes)
\item Encapsulation/passivation release criteria and evidence requirements
\item Handling SOP (carriers, tweezers, storage, labeling)
\item Waste handling and disposal procedure (facility-specific)
\end{enumerate}

\section{Key Concepts (Layperson-Level)}
\subsection{Why ``mobile metals'' are special}
Mobile metals (e.g., Ag) can diffuse under electric fields and during thermal processing. Operationally, the issue is that tiny traces can degrade unrelated processes or create reliability issues for other users (in shared facilities) or for your own future runs (in private toolchains).

\subsection{Why encapsulation is not a magic shield}
Encapsulation reduces surface exposure, but it does not eliminate:
\begin{itemize}[leftmargin=*,itemsep=2pt]
\item wafer breakage/chipping risk,
\item edge exposure risk,
\item pinhole/incomplete coverage risk,
\item later etches or handling that can open encapsulation.
\end{itemize}

\section{Toolchain Segregation (Mandatory)}
\subsection{Define tool zones}
Define explicit zones (names are flexible; the segregation is the point):
\begin{itemize}[leftmargin=*,itemsep=2pt]
\item \textbf{Green zone (clean):} must never see exposed mobile metals.
\item \textbf{Yellow zone (conditionally compatible):} may see encapsulated mobile metals under defined rules.
\item \textbf{Red zone (restricted):} dedicated to mobile-metal work; assume residual risk exists.
\end{itemize}

\subsection{Create a zone transition rule}
A sample may move between zones only if:
\begin{itemize}[leftmargin=*,itemsep=2pt]
\item its sample state label (S0--S3) is current,
\item required encapsulation release criteria are satisfied and documented,
\item the destination tool class permits that state.
\end{itemize}

\section{Step-by-Step Procedure}

\subsection{Step 1: Create a materials registry}
\textbf{Goal:} One source of truth for everything that might be on your samples.

\textbf{Steps}
\begin{enumerate}[leftmargin=*,itemsep=2pt]
\item List every material used in any process step: metals, dielectrics, resists, developers, solvents, adhesion layers, etch chemistries.
\item For each material, record: chemical name, vendor/part number, lot ID, and where it appears in the stack.
\item Flag restricted categories: mobile metals (Ag), noble metals, alkali ions, halogens, organics.
\end{enumerate}

\textbf{Outputs}
\begin{itemize}[leftmargin=*,itemsep=2pt]
\item Materials Registry (versioned)
\item Restricted Materials List (subset)
\end{itemize}

\subsection{Step 2: Define sample state classes}
\textbf{Goal:} A facility-readable label that instantly communicates risk.

\textbf{Recommended states}
\begin{itemize}[leftmargin=*,itemsep=2pt]
\item \textbf{S0:} No restricted materials.
\item \textbf{S1:} Restricted materials present but fully encapsulated per defined criteria.
\item \textbf{S2:} Restricted materials exposed at surface (or potentially exposed at edges).
\item \textbf{S3:} Unknown/uncertain state (treated as most restrictive until cleared).
\end{itemize}

\textbf{Outputs}
\begin{itemize}[leftmargin=*,itemsep=2pt]
\item Sample State Standard v1.0
\item Sample label template (ID, state, date, operator)
\end{itemize}

\subsection{Step 3: Create the tool compatibility matrix}
\textbf{Goal:} Prevent accidental policy violations.

\textbf{Steps}
\begin{enumerate}[leftmargin=*,itemsep=2pt]
\item List tool classes you will use: deposition, lithography, etch/clean, metrology, dicing, packaging.
\item For each class, mark which sample states are permitted.
\item Overlay the zone constraints (Green/Yellow/Red) and define allowed transitions.
\end{enumerate}

\textbf{Output:} Tool Compatibility Matrix (facility-specific variants).

\subsection{Step 4: Define encapsulation/passivation release criteria}
\textbf{Goal:} A testable condition for moving from S2 to S1.

\textbf{Steps}
\begin{enumerate}[leftmargin=*,itemsep=2pt]
\item Define candidate encapsulation layers (facility-compatible) and thickness targets.
\item Define required verification evidence: thickness measurement and inspection images at standard sites.
\item Define a release checklist with sign-off.
\end{enumerate}

\textbf{Output:} Encapsulation Release Checklist (S2$\rightarrow$S1).

\subsection{Step 5: Handling and labeling SOP}
\textbf{Goal:} Make contamination control habitual.

\textbf{Minimum SOP}
\begin{itemize}[leftmargin=*,itemsep=2pt]
\item Dedicated carriers per zone or per state.
\item Dedicated tweezers per restricted class (or facility-approved cleaning procedure).
\item Standard label format: project, sample ID, state (S0--S3), date, operator.
\item Storage segregation: physical separation by state/zone.
\item ``No unlabeled sample'' rule.
\end{itemize}

\textbf{Outputs}
\begin{itemize}[leftmargin=*,itemsep=2pt]
\item Handling SOP
\item Label templates
\item Storage map (where each zone/state is stored)
\end{itemize}

\section{Exit Criteria}
Guide 02 is complete when:
\begin{enumerate}[leftmargin=*,itemsep=2pt]
\item materials registry exists and is versioned,
\item sample-state labels are defined and used consistently,
\item toolchain segregation plan exists (even in private toolchains),
\item tool compatibility matrix exists (per facility/toolchain),
\item encapsulation release criteria are enforceable with evidence,
\item no sample moves between zones without a recorded state label and (if required) release sign-off.
\end{enumerate}

\end{document}
