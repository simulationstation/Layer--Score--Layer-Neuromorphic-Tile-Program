% Standalone LaTeX file (Overleaf-ready)
% Guide 03: Design Toolchain, Layer Maps, and DFM Rules
% Compile with: PDFLaTeX

\documentclass[11pt]{article}

\usepackage[margin=1in]{geometry}
\usepackage{microtype}
\usepackage{xcolor}
\usepackage{hyperref}
\usepackage{booktabs}
\usepackage{longtable}
\usepackage{array}
\usepackage{enumitem}
\usepackage{siunitx}
\usepackage{fancyhdr}

\definecolor{darkblue}{rgb}{0.0, 0.0, 0.5}
\hypersetup{
  colorlinks=true,
  linkcolor=darkblue,
  urlcolor=darkblue,
  citecolor=darkblue,
  pdftitle={Guide 03: Design Toolchain, Layer Maps, and DFM Rules},
  pdfauthor={Open Research Draft}
}

\sisetup{detect-all=true}

\newcommand{\ProgramName}{Layer--Score--Layer Neuromorphic Tile Program}
\newcommand{\GuideID}{Guide 03}
\newcommand{\GuideTitle}{Design Toolchain, Layer Maps, and DFM Rules}
\newcommand{\DocVersion}{v1.1 (standalone)}
\newcommand{\DocDate}{\today}

\setlength{\parindent}{0pt}
\setlength{\parskip}{0.55em}

\pagestyle{fancy}
\fancyhf{}
\lhead{\GuideID: \GuideTitle}
\rhead{\DocVersion}
\cfoot{\thepage}

\newcolumntype{P}[1]{>{\raggedright\arraybackslash}p{#1}}

\begin{document}

\begin{titlepage}
  \centering
  \vspace*{1.0in}
  {\LARGE \GuideID: \GuideTitle\par}
  \vspace{0.25in}
  {\Large \ProgramName\par}
  \vspace{0.35in}
  {\large \DocVersion\par}
  \vspace{0.10in}
  {\large \DocDate\par}
  \vfill
  \begin{minipage}{0.93\textwidth}
  \small
  \textbf{Purpose.} Standardize the design-to-fabrication interface: CAD toolchain, layer maps, DFM/DRC rules, alignment strategy, and standard test structures. \textbf{Updated emphasis:} pad/probing/packaging survivability requirements are design-time constraints (not a late-stage integration detail).
  \end{minipage}
  \vspace*{0.6in}
\end{titlepage}

\tableofcontents
\newpage

\section{Core Deliverables}
\begin{enumerate}[leftmargin=*,itemsep=2pt]
\item Design repository structure and versioning policy
\item Layer map (design layers \(\rightarrow\) fabricated layers) and naming conventions
\item Alignment mark strategy (global/die/local) and standard vernier structures
\item Standard test structure library (devices + metrology structures)
\item DFM/DRC rule set per patterning tier (printed, maskless UV, scoring)
\item Tapeout checklist and first-article protocol
\item Padout and probing/packaging design rules (required for packaging readiness)
\end{enumerate}

\section{Step-by-Step Procedure}
\begin{enumerate}[leftmargin=*,itemsep=2pt]
\item Select CAD toolchain and define the ``source of truth'' file format (e.g., GDS/OASIS).
\item Define repository structure (recommended: \texttt{/cad}, \texttt{/process}, \texttt{/test}, \texttt{/data}, \texttt{/models}, \texttt{/governance}).
\item Create and freeze a project layer map early; treat changes as governance events.
\item Define alignment marks and placement:
  \begin{itemize}[leftmargin=*,itemsep=2pt]
  \item global/wafer marks (coarse alignment)
  \item die marks (medium alignment)
  \item local marks near critical devices (fine alignment)
  \end{itemize}
\item Create a standard library of test structures:
  \begin{itemize}[leftmargin=*,itemsep=2pt]
  \item isolated single devices
  \item device arrays for statistical characterization
  \item line resistance monitors and continuity chains
  \item Kelvin structures for contact resistance
  \item overlay verniers and CD monitors
  \end{itemize}
\item Define DFM/DRC rules per tier (min line/space, enclosures, via sizes, keep-outs).
\item Create a tapeout checklist and enforce it for every run.
\end{enumerate}

\section{Padout and Probing/Packaging Rules (High Priority)}
Packaging is a common late-stage failure mechanism. Therefore, padout and probing rules are required during layout.

\subsection{Pad design checklist (minimum)}
\begin{itemize}[leftmargin=*,itemsep=2pt]
\item \textbf{Pad geometry:} pad size and pitch must match intended probe tips and/or wirebond process.
\item \textbf{Pad metallurgy:} pad surface must be probeable/bondable under the chosen process window.
\item \textbf{Underlayer robustness:} avoid pad stacks too thin/soft for bonding force.
\item \textbf{Keep-outs:} keep sensitive devices away from pad edge stress regions and from dicing streets.
\item \textbf{Coupon region:} include a bond/probe coupon region with redundant pads and simple continuity structures.
\end{itemize}

\subsection{Early validation requirement}
Before attempting full device arrays, the program requires at least one first-article validation of:
\begin{itemize}[leftmargin=*,itemsep=2pt]
\item pad survivability under probing and/or bonding,
\item pad-to-routing continuity,
\item susceptibility to oxidation/damage under handling.
\end{itemize}

\section{Exit Criteria}
Guide 03 is complete when a third party can take your \texttt{/cad} package, your layer map, your DFM rules, and your padout rules and generate a facility-ready file set without asking what any layer means or whether pads are probe/bond compatible.

\end{document}
