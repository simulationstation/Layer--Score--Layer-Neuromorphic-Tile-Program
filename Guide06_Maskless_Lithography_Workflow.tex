% Standalone LaTeX file (Overleaf-ready)
% Guide 06: Maskless Lithography Workflow
% Compile with: PDFLaTeX

\documentclass[11pt]{article}

\usepackage[margin=1in]{geometry}
\usepackage{microtype}
\usepackage{xcolor}
\usepackage{hyperref}
\usepackage{booktabs}
\usepackage{longtable}
\usepackage{array}
\usepackage{enumitem}
\usepackage{siunitx}
\usepackage{fancyhdr}

\definecolor{darkblue}{rgb}{0.0, 0.0, 0.5}
\hypersetup{
  colorlinks=true,
  linkcolor=darkblue,
  urlcolor=darkblue,
  citecolor=darkblue,
  pdftitle={Guide 06: Maskless Lithography Workflow},
  pdfauthor={Open Research Draft}
}

\sisetup{detect-all=true}

\newcommand{\ProgramName}{Layer--Score--Layer Neuromorphic Tile Program}
\newcommand{\GuideID}{Guide 06}
\newcommand{\GuideTitle}{Maskless Lithography Workflow}
\newcommand{\DocVersion}{v1.1 (standalone)}
\newcommand{\DocDate}{\today}

\setlength{\parindent}{0pt}
\setlength{\parskip}{0.55em}

\pagestyle{fancy}
\fancyhf{}
\lhead{\GuideID: \GuideTitle}
\rhead{\DocVersion}
\cfoot{\thepage}

\newcolumntype{P}[1]{>{\raggedright\arraybackslash}p{#1}}

\begin{document}

\begin{titlepage}
  \centering
  \vspace*{1.0in}
  {\LARGE \GuideID: \GuideTitle\par}
  \vspace{0.25in}
  {\Large \ProgramName\par}
  \vspace{0.35in}
  {\large \DocVersion\par}
  \vspace{0.10in}
  {\large \DocDate\par}
  \vfill
  \begin{minipage}{0.93\textwidth}
  \small
  \textbf{Purpose.} Make maskless optical lithography the throughput backbone with repeatable alignment and predictable pattern quality. This guide defines standard resist stacks, alignment verification, exposure logging, diagnostics, and first-article procedure.
  \end{minipage}
  \vspace*{0.6in}
\end{titlepage}

\tableofcontents
\newpage

\section{Core Deliverables}
\begin{itemize}[leftmargin=*,itemsep=2pt]
\item Standard resist stacks for (a) metal lift-off and (b) dielectric etch patterns.
\item Bake/develop workflow with acceptable process windows and logging requirements.
\item Alignment procedure and verification method (verniers) and standard image capture sites.
\item Exposure parameter logging and re-use rules.
\item Failure mode atlas (scumming, footing, undercut loss, edge bead, residues).
\item First-article procedure for each new design revision.
\end{itemize}

\section{Layperson Overview (What matters and why)}
Maskless lithography is your ``factory line'' for prototypes: it controls which parts of each layer exist and where they land. Most downstream failures that look like ``memristor physics'' are actually caused by:
\begin{itemize}[leftmargin=*,itemsep=2pt]
\item misalignment between layers (overlay error),
\item incomplete development or residue (creating shorts/leaks),
\item poor lift-off profile (creating fences),
\item topography that exceeds focus margin.
\end{itemize}
This guide standardizes the workflow so you are not reinventing a process window every run.

\section{Step-by-Step Procedure}

\subsection{Step 1: Define standard resist stacks}
\textbf{Goal:} avoid per-layer improvisation.
\begin{enumerate}[leftmargin=*,itemsep=2pt]
\item Choose a default lift-off resist strategy and a default etch resist strategy compatible with facility tools.
\item Define bake times/temperatures and developer type/time as controlled variables.
\item Define ``do not change without justification'' parameters for the next two to three runs.
\end{enumerate}

\subsection{Step 2: Alignment strategy and mark usage}
\textbf{Goal:} reproducible overlay.
\begin{enumerate}[leftmargin=*,itemsep=2pt]
\item Use multi-scale alignment marks (global, die, local).
\item Define which marks are used for which layers.
\item Define vernier structures and locations for overlay measurement.
\item Require image capture at standard vernier sites after alignment.
\end{enumerate}

\subsection{Step 3: Exposure logging (non-negotiable)}
\begin{enumerate}[leftmargin=*,itemsep=2pt]
\item Record exposure mode, dose/energy settings, write strategy, focus settings (if applicable), and pattern file version/hash.
\item Record deviations: if you changed dose or focus mid-run, log it immediately.
\item Attach key images (alignment screen captures, post-develop microscope images).
\end{enumerate}

\subsection{Step 4: Post-pattern inspection gates}
\textbf{Goal:} detect preventable failures early.
\begin{enumerate}[leftmargin=*,itemsep=2pt]
\item Define ``must inspect'' sites and features (pads, critical device regions, verniers).
\item Define what triggers rework (gross residue, collapsed features, major overlay error).
\item If a gate fails, stop and document; do not proceed to irreversible depositions.
\end{enumerate}

\subsection{Step 5: Diagnostics atlas (build as you go)}
\begin{enumerate}[leftmargin=*,itemsep=2pt]
\item Maintain a reference set of failure photos with likely causes.
\item Link each failure mode to a decision: rework, scrap, or proceed.
\end{enumerate}

\subsection{Step 6: First-article procedure}
\textbf{Goal:} avoid surprises when designs change.
\begin{enumerate}[leftmargin=*,itemsep=2pt]
\item For each new design revision, run a first-article check on a small subset.
\item Verify alignment, critical dimensions (where measurable), and pad integrity.
\item Only then proceed to full-run execution.
\end{enumerate}

\section{Exit Criteria}
Guide 06 is complete when the same operator can produce repeatable patterns and overlay (within program needs) on two different days without ad-hoc parameter hunting, and when exposure/alignment logs are sufficient for another operator to reproduce the run.

\end{document}
