% Standalone LaTeX file (Overleaf-ready)
% Guide 14: Packaging, Board Integration, and Tile Demonstration
% Compile with: PDFLaTeX

\documentclass[11pt]{article}

\usepackage[margin=1in]{geometry}
\usepackage{microtype}
\usepackage{xcolor}
\usepackage{hyperref}
\usepackage{booktabs}
\usepackage{longtable}
\usepackage{array}
\usepackage{enumitem}
\usepackage{siunitx}
\usepackage{fancyhdr}

\definecolor{darkblue}{rgb}{0.0, 0.0, 0.5}
\hypersetup{
  colorlinks=true,
  linkcolor=darkblue,
  urlcolor=darkblue,
  citecolor=darkblue,
  pdftitle={Guide 14: Packaging, Board Integration, and Tile Demonstration},
  pdfauthor={Open Research Draft}
}

\sisetup{detect-all=true}

\newcommand{\ProgramName}{Layer--Score--Layer Neuromorphic Tile Program}
\newcommand{\GuideID}{Guide 14}
\newcommand{\GuideTitle}{Packaging, Board Integration, and Tile Demonstration}
\newcommand{\DocVersion}{v1.1 (standalone)}
\newcommand{\DocDate}{\today}

\setlength{\parindent}{0pt}
\setlength{\parskip}{0.55em}

\pagestyle{fancy}
\fancyhf{}
\lhead{\GuideID: \GuideTitle}
\rhead{\DocVersion}
\cfoot{\thepage}

\newcolumntype{P}[1]{>{\raggedright\arraybackslash}p{#1}}

\begin{document}

\begin{titlepage}
  \centering
  \vspace*{1.0in}
  {\LARGE \GuideID: \GuideTitle\par}
  \vspace{0.25in}
  {\Large \ProgramName\par}
  \vspace{0.35in}
  {\large \DocVersion\par}
  \vspace{0.10in}
  {\large \DocDate\par}
  \vfill
  \begin{minipage}{0.93\textwidth}
  \small
  \textbf{Purpose.} Deliver system-level demonstrators that survive outside the probe station, with credible measurement boundaries and repeatable harnesses. This guide covers packaging mode choices, board-level integration, benchmark runners, and energy measurement boundaries.
  \\
  \\
  \textbf{Key dependency.} Packaging readiness begins at design time (Guide 03 pad/probing rules). Probe-only success is not sufficient for tile claims.
  \end{minipage}
  \vspace*{0.6in}
\end{titlepage}

\tableofcontents
\newpage

\section{Core Deliverables}
\begin{enumerate}[leftmargin=*,itemsep=2pt]
\item Packaging selection guide by stage (probe, wirebond breakout, flip-chip/interposer)
\item ESD and handling SOP for packaged parts
\item Standard test board (power rails, biasing, measurement headers)
\item Firmware/software harness for benchmarks and structured logging
\item Energy measurement boundary statement (intrinsic device vs system-level)
\item Packaging first-article coupon plan (pad survivability validation)
\end{enumerate}

\section{Stage-Appropriate Packaging Choices}
\subsection{Probe-first (early phases)}
\begin{itemize}[leftmargin=*,itemsep=2pt]
\item Best for P0--P1 bring-up and rapid iteration.
\item Requires pad design compatible with repeated probing (Guide 03).
\item Risk: probe-induced damage and pad oxidation can create false ``device drift.''
\end{itemize}

\subsection{Wirebond breakout (mid phases)}
\begin{itemize}[leftmargin=*,itemsep=2pt]
\item Best for P2 arrays and early P3 tile harnesses.
\item Requires pad metallurgy compatible with the chosen wirebond process window.
\item Risk: yield loss at dicing/wirebond is common; validate with coupons.
\end{itemize}

\subsection{Flip-chip / interposer (later phases)}
\begin{itemize}[leftmargin=*,itemsep=2pt]
\item Best when I/O density and parasitics dominate performance.
\item Higher cost and longer cycle time; typically not first choice.
\end{itemize}

\section{Packaging First-Article Requirement (Non-Optional)}
\textbf{Requirement:} include a bond/probe coupon region in early designs to validate:
\begin{itemize}[leftmargin=*,itemsep=2pt]
\item pad metallization survivability under intended probe/bond process,
\item pad-to-routing continuity,
\item susceptibility to oxidation/damage under handling.
\end{itemize}

\section{Standard Test Board and Harness}
\subsection{Board requirements}
\begin{itemize}[leftmargin=*,itemsep=2pt]
\item Define power rails and bias rails; include measurement points for each.
\item Include current sense points or shunts where appropriate.
\item Provide shielding/grounding strategy appropriate to pulse measurements.
\item Maintain a stable board revision across runs unless a formal revision is logged.
\end{itemize}

\subsection{Harness requirements}
\begin{itemize}[leftmargin=*,itemsep=2pt]
\item Scripted benchmark runner (no manual ``demo mode'').
\item Structured logging: run ID, device/tile ID, firmware version, environment.
\item Automated plotting and report generation.
\end{itemize}

\section{Energy Measurement Boundaries (Discipline)}
\subsection{Two boundaries to report when possible}
\begin{itemize}[leftmargin=*,itemsep=2pt]
\item \textbf{Intrinsic device energy:} tight boundary around the device event.
\item \textbf{System energy per task:} measured at module input rails, from task-start to task-done.
\end{itemize}

\subsection{Boundary statement (required in every demo report)}
Each demo report must include:
\begin{itemize}[leftmargin=*,itemsep=2pt]
\item what rails are included in energy integration,
\item what is excluded and why,
\item whether idle/baseline is subtracted (and how).
\end{itemize}

\section{Procedure (End-to-End)}
\begin{enumerate}[leftmargin=*,itemsep=2pt]
\item Select packaging mode appropriate to the current phase and document it.
\item Validate pads using coupon structures before packaging full arrays.
\item Assemble packaged parts; apply ESD handling discipline.
\item Bring up the standard test board and verify rail integrity.
\item Run the benchmark runner end-to-end; store logs with run IDs matching the fabrication traveler.
\item Report results with the defined energy boundary statement and confidence intervals where appropriate.
\end{enumerate}

\section{Exit Criteria}
Guide 14 is complete when:
\begin{enumerate}[leftmargin=*,itemsep=2pt]
\item a packaged tile or module can run a benchmark end-to-end using a reproducible harness,
\item energy per task is reported with an explicit boundary definition,
\item a third party can reproduce the demo using your board files, firmware/scripts, and documentation.
\end{enumerate}

\end{document}
